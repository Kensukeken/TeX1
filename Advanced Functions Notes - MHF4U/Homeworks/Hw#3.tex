\documentclass{article}
\usepackage[utf8]{inputenc}
\usepackage[T1]{fontenc}
\usepackage{graphicx}
\usepackage{amsmath, amssymb}
\usepackage{xcolor}
\usepackage{tikz}
\usepackage{lipsum}
\usepackage{sectsty}

\title{Practice Questions}
\author{Kensukeken}
\date{February 27th, 2024}
\begin{document}
\maketitle
\subsubsection*{Chapter 1.5 \& 1.6}
\section*{Page 49 \#1-5, 7, 8, 10, 14}
\begin{itemize}

\item Question 1:
\subsubsection*{Parameters and Transformations}
\begin{itemize}
    \item The given equation is \(y = 4[3(x + 2)]^4 - 6\).
    \item Let's break down the transformations:
    \begin{itemize}
        \item \textbf{Vertical Stretch}: The coefficient \(4\) stretches the graph vertically.
        \item \textbf{Horizontal Compression}: The factor \(3\) compresses the graph horizontally.
        \item \textbf{Horizontal Shift}: The \(+2\) inside the parentheses shifts the graph to the left by \(2\) units.
        \item \textbf{Vertical Shift}: The \(-6\) shifts the graph downward by \(6\) units.
    \end{itemize}
\end{itemize}

\subsubsection*{Table Completion}
\begin{itemize}
    \item We're given some points for the original equation \(y = x^4\):
    \begin{itemize}
        \item \((-2, 16)\)
        \item \((-1, 1)\)
        \item \((0, 0)\)
        \item \((1, 1)\)
        \item \((2, 16)\)
    \end{itemize}
\end{itemize}

\subsubsection*{Graph Sketching}
\begin{itemize}
    \item To sketch the graph of \(y = 4[3(x + 2)]^4 - 6\), apply the transformations to the graph of \(y = x^4\).
    \item Start with the original graph and adjust it based on the transformations.
\end{itemize}

\subsubsection*{Additional Information}
\begin{itemize}
    \item \textbf{Domain}: The domain remains all real numbers.
    \item \textbf{Range}: The range is \((-6, \infty)\).
    \item \textbf{Vertex}: The vertex occurs at \((-2, -6)\).
    \item \textbf{Axis of Symmetry}: The axis of symmetry is the vertical line \(x = -2\).
\end{itemize}
\item Question 2:
\begin{enumerate}
    \item[a)] \(y = -x^n\) corresponds to \textbf{ii) reflection in the x-axis}.
    \item[b)] \(y = (-x)^n + 2\) corresponds to \textbf{iv) reflection in the y-axis}, because it's a vertical shift upwards.
    \item[c)] \(y = -(-x)^n\) corresponds to \textbf{iii) reflection in both the x-axis and the y-axis}.
    \item[d)] \(y = x^n\) corresponds to \textbf{i) no reflection}.
\end{enumerate}

\item Question 3:
\begin{enumerate}
    \item \(y = 2x^n\) corresponds to \textbf{iii) vertically stretched by a factor of 2}.
    \item \(y = (2x)^n\) corresponds to \textbf{i) horizontally stretched by a factor of 2}.
    \item \(y = \frac{1}{2}x^n\) corresponds to \textbf{ii) vertically compressed by a factor of \(\frac{1}{2}\)}.
    \item \(y = \left(\frac{1}{2}x\right)^n\) corresponds to \textbf{iv) horizontally compressed by a factor of \(\frac{1}{2}\)}.
\end{enumerate}

\item Question 4:
\subsubsection*{Function (a)}
\begin{itemize}
    \item Equation: \(y = (3x)^3 - 1\)
    \item Parameters:
    \begin{itemize}
        \item \(a = 1\) (since there's no coefficient in front of the base function)
        \item \(k = 3\) (from the exponent inside the parentheses)
        \item \(d = 0\) (no shift in the \(x\)-direction)
        \item \(c = -1\) (constant term)
    \end{itemize}
    \item Degree: \(n = 3\)
    \item Transformation: Cubic function with a vertical shift downward by 1 unit.
\end{itemize}

\subsubsection*{Function (b)}
\begin{itemize}
    \item Equation: \(y = 0.4(x + 2)^2\)
    \item Parameters:
    \begin{itemize}
        \item \(a = 0.4\)
        \item \(k = 1\) (since the base function is \((x + 2)^2\))
        \item \(d = -2\) (shift to the left by 2 units)
        \item \(c = 0\) (no vertical shift)
    \end{itemize}
    \item Degree: \(n = 2\)
    \item Transformation: Quadratic function with a horizontal shift to the left by 2 units.
\end{itemize}

\subsubsection*{Function (c)}
\begin{itemize}
    \item Equation: \(y = x^3 + 5\)
    \item Parameters:
    \begin{itemize}
        \item \(a = 1\) (no coefficient in front of the base function)
        \item \(k = 1\) (base function is \(x^3\))
        \item \(d = 0\) (no shift in the \(x\)-direction)
        \item \(c = 5\) (vertical shift upward by 5 units)
    \end{itemize}
    \item Degree: \(n = 3\)
    \item Transformation: Cubic function with a vertical shift upward by 5 units.
\end{itemize}

\subsubsection*{Function (d)}
\begin{itemize}
    \item Equation: \(y = \frac{3}{4}[-(x - 4)]^3 + 1\)
    \item Parameters:
    \begin{itemize}
        \item \(a = \frac{3}{4}\)
        \item \(k = 1\) (base function is \(-(x - 4)^3\))
        \item \(d = 4\) (shift to the right by 4 units)
        \item \(c = 1\) (vertical shift upward by 1 unit)
    \end{itemize}
    \item Degree: \(n = 3\)
    \item Transformation: Cubic function with a horizontal shift to the right by 4 units and a vertical shift upward by 1 unit.
\end{itemize}

\subsubsection*{Function (e)}
\begin{itemize}
    \item Equation: \(y = 2\left(\frac{1}{3}x\right)^4 - 5\)
    \item Parameters:
    \begin{itemize}
        \item \(a = 2\)
        \item \(k = \frac{1}{3}\)
        \item \(d = 0\) (no shift in the \(x\)-direction)
        \item \(c = -5\) (vertical shift downward by 5 units)
    \end{itemize}
    \item Degree: \(n = 4\)
    \item Transformation: Quartic function with a vertical shift downward by 5 units.
\end{itemize}

\subsubsection*{Function (f)}
\begin{itemize}
    \item Equation: \(y = 8[(2x)^3] + 3\)
    \item Parameters:
    \begin{itemize}
        \item \(a = 8\)
        \item \(k = 1\) (base function is \((2x)^3\))
        \item \(d = 0\) (no shift in the \(x\)-direction)
        \item \(c = 3\) (vertical shift upward by 3 units)
    \end{itemize}
    \item Degree: \(n = 3\)
    \item Transformation: Cubic function with no horizontal shift and a vertical shift upward by 3 units.
\end{itemize}
\item Question 5:
\subsubsection*{Function (i)}
\begin{itemize}
    \item Equation: \(y = -\frac{1}{x^3}\)
    \item Description:
    \begin{itemize}
        \item This function is a reciprocal cubic function.
        \item As \(x\) approaches positive or negative infinity, \(y\) approaches 0.
        \item The graph should have a vertical asymptote at \(x = 0\).
    \end{itemize}
    \item Justification: Graph (c) seems to fit this description. It has a vertical asymptote at \(x = 0\), and its shape resembles a reciprocal cubic curve.
\end{itemize}

\subsubsection*{Function (ii)}
\begin{itemize}
    \item Equation: \(y = x^2 - 1\)
    \item Description:
    \begin{itemize}
        \item This function is a quadratic function.
        \item The graph should open upward (since the coefficient of \(x^2\) is positive) and intersect the \(y\)-axis at \(y = -1\).
    \end{itemize}
    \item Justification: Graph (b) matches this description. It's a parabolic curve opening upward and intersects the \(y\)-axis at \(y = -1\).
\end{itemize}

\subsubsection*{Function (iii)}
\begin{itemize}
    \item Equation: \(y = -\left(\frac{1}{x}\right)^5\)
    \item Description:
    \begin{itemize}
        \item This function is a reciprocal quintic function.
        \item As \(x\) approaches positive or negative infinity, \(y\) approaches 0.
        \item The graph should have a vertical asymptote at \(x = 0\).
    \end{itemize}
    \item Justification: Graph (d) fits this pattern. It has a vertical asymptote at \(x = 0\) and resembles a reciprocal quintic curve.
\end{itemize}

\subsubsection*{Function (iv)}
\begin{itemize}
    \item Equation: \(y = -x^4\)
    \item Description:
    \begin{itemize}
        \item This function is a quartic function.
        \item The graph should open downward (since the coefficient of \(x^4\) is negative).
    \end{itemize}
    \item Justification: Graph (a) aligns with this description. It's a downward-opening curve that resembles a quartic function.
\end{itemize}
\item Question 7:
\subsubsection*{a)}
The parameters of the polynomial function \( y = -3\left(\frac{1}{2}\right)(x + 4)^{14} + 1 \) are:
\begin{itemize}
    \item \textbf{Vertical stretch/compression and reflection}: The coefficient \(-3\left(\frac{1}{2}\right)\)
    \item \textbf{Horizontal shift}: The term \((x + 4)\)
    \item \textbf{Exponent}: The power of \(14\)
    \item \textbf{Vertical shift}: The constant \(+1\)
\end{itemize}

\subsubsection*{b)}
Each parameter transforms the base graph \( y = x^4 \) as follows:
\begin{itemize}
    \item The coefficient \(-3\left(\frac{1}{2}\right)\) reflects the graph across the x-axis and compresses it vertically by a factor of \(\frac{3}{2}\).
    \item The term \((x + 4)\) shifts the graph 4 units to the left.
    \item The exponent \(14\) makes the graph steeper compared to the base graph.
    \item The constant \(+1\) shifts the graph 1 unit upwards.
\end{itemize}

\subsubsection*{c)}
For the transformed function, the:
\begin{itemize}
    \item \textbf{Domain} is all real numbers, \( (-\infty, \infty) \).
    \item \textbf{Range} is \( [1, \infty) \) due to the vertical shift up by 1.
    \item \textbf{Vertex} is at \( (-4, 1) \), resulting from the horizontal shift left by 4 and vertical shift up by 1.
    \item \textbf{Axis of symmetry} is the vertical line \( x = -4 \).
\end{itemize}

\subsubsection*{d)}
Two possible orders of transformations are:
\begin{enumerate}
    \item Start with a horizontal shift, followed by a reflection and vertical stretch/compression, and finally a vertical shift.
    \item Begin with a reflection and vertical stretch/compression, then apply the horizontal shift, and end with the vertical shift.
\end{enumerate}
\item Question 8:
\subsubsection*{a)}
For \( f(x) = x^3 \) and \( y = -0.5(f(x - 4)) \), the transformations are:
\begin{itemize}
    \item \textbf{Horizontal shift}: 4 units to the right (because of \( x - 4 \)).
    \item \textbf{Vertical stretch/compression}: Compressed by a factor of 0.5 (because of the coefficient \( -0.5 \)).
    \item \textbf{Reflection}: Reflected over the x-axis (because the coefficient is negative).
\end{itemize}
The full equation after transformation is \( y = -0.5(x - 4)^3 \).

\subsubsection*{b)}
For \( f(x) = x^4 \) and \( y = -(f(4x)) + 1 \), the transformations are:
\begin{itemize}
    \item \textbf{Horizontal stretch/compression}: Compressed horizontally by a factor of 1/4 (because of \( 4x \)).
    \item \textbf{Reflection}: Reflected over the x-axis (because of the negative sign before the function).
    \item \textbf{Vertical shift}: 1 unit upwards (because of \( + 1 \)).
\end{itemize}
The full equation after transformation is \( y = -(4x)^4 + 1 \).

\subsubsection*{c)}
For \( f(x) = x^3 \) and \( y = 2f\left(\frac{1}{3}(x - 5)\right) - 2 \), the transformations are:
\begin{itemize}
    \item \textbf{Horizontal stretch/compression}: Stretched horizontally by a factor of 3 (because of \( \frac{1}{3} \)).
    \item \textbf{Horizontal shift}: 5 units to the right (because of \( x - 5 \)).
    \item \textbf{Vertical stretch/compression}: Stretched vertically by a factor of 2 (because of the coefficient \( 2 \)).
    \item \textbf{Vertical shift}: 2 units downwards (because of \( - 2 \)).
\end{itemize}
The full equation after transformation is \( y = 2\left(\frac{1}{3}(x - 5)\right)^3 - 2 \).

\item Question 14:
\subsubsection*{a)}
The graph of \( y = x^3 - x^2 \) is a cubic function with a turning point at the origin (0,0), where the function changes from increasing to decreasing or vice versa. The graph of \( y' = (x - 2)^3 - (x - 2)^2 \) is a horizontal shift of the original function 2 units to the right. This is because each \( x \) in the original function is replaced with \( x - 2 \), which translates the graph to the right.

\subsubsection*{b)}
To verify the prediction, you can graph both functions using graphing technology like a graphing calculator or software. You should see that the second graph is indeed the first graph shifted to the right by 2 units.

\subsubsection*{c)}
To find the x-intercepts of the original function, you can factor the equation:
\[ y = x^2(x - 1) \]
The x-intercepts are found when \( y = 0 \), which gives us \( x = 0 \) and \( x = 1 \).

For the transformed function, you can expand and then factor the equation:
\[ y' = (x - 2)^2(x - 3) \]
The x-intercepts for this function are \( x = 2 \) and \( x = 3 \).

\end{itemize}


\section*{Page 62 \#1-5, 7, 8, 9, 12}
\begin{itemize}
\item Question 1:
\begin{enumerate}
    \item[(a)] A child grows $8 \, \text{cm}$ in $6$ months: This involves a rate of change as the child's height is changing over time.
    \item[(b)] The temperature at a $750$-m-high ski hill is $2^\circ C$ at the base and $-8^\circ C$ at the top: This involves a rate of change as temperature changes with altitude.
    \item[(c)] A speedometer shows that a vehicle is traveling at $90 \, \text{km/h}$: This involves a rate of change as it indicates speed, which is distance over time.
    \item[(d)] A jogger ran $23 \, \text{km}$ in $2$ hours: This also involves a rate of change as it indicates speed.
    \item[(e)] The laptop cost \$750: This option does not involve any rate of change. It's just stating the price of an item without any variation over time or other variables.
    \item[(f)] A plane traveled $650 \, \text{km}$ in $3$ hours: This involves a rate of change as it indicates speed.
\end{enumerate}

Therefore, the correct answer is \textbf{(e)}, as it does not represent an average rate of change. The laptop's cost is a static value and does not involve change over time.

\item Question 2:
\begin{enumerate}
    \item[2a)] The rate of change of a straight line remains constant. Whether we're considering the instantaneous rate of change or the average rate of change, it all boils down to finding the slope of the line. So, let's take a line passing through the points (0, 24) and (2, 30). The rise here is 6 and the run is 2, yielding a slope of 3. Hence, this slope represents both the instantaneous and average rate of change.
    
    \item[2b)] Consider a straight horizontal line. The slope of a line is defined by the ratio of the rise to the run. For a horizontal line, no matter how far we traverse, the rise remains zero. Consequently, the rate of change at any point, as well as the average rate of change, is zero for horizontal lines.
    
    \item[2c)] The average rate of change of any linear function is simply the slope of the line representing it. Let's take the points (0, 6) and (10, 6) to determine the slope. The rise is 0 and the run is 10, giving us a slope of 0. Therefore, the average rate of change for this linear function is 0.
\end{enumerate}

\item Question 4:
The average rate of change can be calculated as the percentage change divided by the number of years. Let's calculate it for the given scenario:

The change in percentage is $66.8\% - 16.2\% = 50.6\%$. The number of years is $2003 - 1990 = 13$ years.

Therefore, the average rate of change, indicating the increase in the number of people with computers in their homes, is $\frac{50.6\%}{13 \text{ years}} \approx 3.89\%$ per year.

\item Question 5:
\begin{enumerate}
    \item[5a)] Determine the average rate of change of the percentage of households using email from 1999 to 2003. We calculate this by dividing the change in percentage by the change in the number of years. For Part A, it's going to be the percentage, $52.1\% - 26.3\%$, divided by $2003 - 1999$. So, it's $ \frac{52.1\% - 26.3\%}{2003 - 1999}$. Punching this into a calculator, we get $6.45\%$ per year. 
    \item[5b)] Why might someone want to know the average rate of change rather than the specific change each year? Well, if you want to understand the general trend — whether something is increasing or decreasing over time, like the adoption of email by households — the average rate of change provides a single number that summarizes this trend. 

In this case, the average rate of change tells us that approximately 6.5 % of people are adopting email every year, indicating that email is becoming more popular over time. This single number offers a quick and clear understanding of the speed at which this trend is occurring.
    \item[5cde)] For Part C, we want to find out the rate of change for every year. All we need to do is subtract the two numbers, because the difference in the years is just one. Set this all in the calculator, and this is the percentage of growth every year. It looks like it's sort of decreasing, then picking up again.

For Part D, if you compare this with the average rate of change calculated earlier, which is about 6.45%, it matches. But knowing the specific numbers tells you whether the adoption of email is actually getting faster or slower. You can see it's getting a bit slower by knowing the specific numbers, but overall, it's still increasing.

\end{enumerate}
\newpage
\item Question 7:
\begin{enumerate}
    \item[7a)]Both volume and surface area are functions of \( R \). Volume is a function of \( R \), and the variable \( R \) is raised to the power of three. That means this is a cubic polynomial, which is a cubic power function. The surface area is also a function of \( R \), and the variable \( R \) is squared. So, it's essentially a quadratic, a parabola. Okay, so this is a degree two power function.

The domain and range of any polynomial is all real values. But because we're talking about volume, the domain of the volume function is the set of real values such that \( R \) is actually greater than or equal to zero. If it's equal to zero, it has no volume, so it should be greater than or equal to zero. The range of the volume function is also greater than or equal to zero.

For the surface area, the domain of the surface area function, again, is the set of real numbers, but \( R \) is greater than or equal to zero. As a result, the range of the surface area function is the set of real values and can be greater than or equal to zero as well.
 \item[7b)]We need to find the average rate of change as the radius decreases from 30 to 25 for both volume and surface area. Let's start with volume:

From 30 to 25, it's \(\frac{{V_{25} - V_{30}}}{{25 - 30}}\). So, we simply plug 25 and 30 into \(\frac{4}{3} \pi R^3\). 

Upon calculation (without decimals for now), it simplifies to a value. But remember, this is actually the rate at which the volume is decreasing, per centimeter.

Similarly, from 25 to 20:

\(\frac{{V_{20} - V_{25}}}{{20 - 25}}\). This tells us how much the volume is decreasing per centimeter. Note that it's not decreasing as fast when the volume is a bit smaller than when it is larger.

Now, let's calculate for the surface area:

\(\frac{{S_{25} - S_{30}}}{{25 - 30}}\). This will tell us how much it's decreasing by. It's \(4 \pi 25^2 - 4 \pi 30^2\) over 5. 

Similarly, from 20 to 25:

\(\frac{{S_{20} - S_{25}}}{{20 - 25}}\). Again, it's just the difference of squares of \(25^2\) and \(30^2\), which is 25 - 30 and \(25 + 30\) cancel out. This simplifies to \(180 \pi\) per centimeter. So, even the surface area is decreasing slower when the radius is smaller.
 \item[7c)]To determine the average rate of change of the surface area when it decreases from one value to another, we use the formula \(\frac{\Delta S}{\Delta R}\). 

So, we find \(\Delta S\) by subtracting the initial surface area from the final surface area: \(1256.64 - 287.43\). 

Now, what's the corresponding change in radius? Well, the surface area formula is \(4 \pi r^2\), and this corresponds to the given surface areas. 

We can rearrange this formula to solve for \(r\): \(r_1 = \sqrt{\frac{1256.64}{4 \pi}}\) for the initial surface area, and \(r_2 = \sqrt{\frac{1256.64}{4 \pi}}\) for the final surface area.

Now, we can calculate the change in radius: \(r_2 - r_1\). 

Substituting the values, we get: 

\(\frac{1256.64 - 287.43}{r_2 - r_1}\). 

This gives us the rate at which the surface area is decreasing per centimeter as the radius decreases.
 \item[7d)]The average rate of change of the volume, we use the formula \(\frac{\Delta V}{\Delta R}\). 

The change in volume (\(\Delta V\)) is calculated by subtracting the initial volume from the final volume: \(942.48 - 1675.52\). 

Now, we need to find the change in radius (\(R_2 - R_1\)). We know that \(\frac{4}{3}\pi R^3\) represents the volume. So, when \(R = R_2\), \((R_2)^3 = 942.48\). Solving for \(R_2\), we get: \(R_2 = \sqrt[3]{\frac{3 \times 942.48}{4 \pi}}\), which simplifies to approximately 6.0.

Similarly, when \(R = R_1\), \((R_1)^3 = 1675.52\). Solving for \(R_1\), we get: \(R_1 = \sqrt[3]{\frac{3 \times 1675.52}{4 \pi}}\), which simplifies to approximately 7.37.

Now, we can substitute these values into the formula: \(\frac{942.48 - 1675.52}{6.0 - 7.37}\). 

Upon calculation, we find that on average, the volume decreases by approximately 568.25 cubic centimeters.
 
\end{enumerate}
\item Question 8:
A cyclist has a constant speed. We're trying to sketch a graph of the cyclist's speed versus time. Initially, it's going at a constant positive speed. As it reaches the top of the hill, the cyclist slows down, decelerating. Then, at the bottom of the hill, it decelerates even more. When it reaches the top of the next hill, it starts climbing, accelerating again, and then goes back to the same constant speed while riding along. 

But when it comes to the next hill, instead of climbing, it stops abruptly. So, the speed becomes zero. 

The graph for speed versus time for this cyclist would look something like this.

\item Question 9:
\textbf{Part A:} We know that the equation of the parabola is of the form \(y = ax(x - 28)\). To find the value of \(a\), we can use the point \((10, 36)\) on the parabola. Substituting these values into the equation, we get \(36 = a \cdot (10 \cdot 10 - 28)\). Solving for \(a\), we find \(a = -\frac{1}{5}\). So, the equation of the parabola is \(y = -\frac{1}{5}x(x - 28)\).

\textbf{Part B:} The line segments on the graph represent lines with slopes. We can estimate the slopes by choosing two points on each line segment and calculating the slope between them.

\textbf{Part C:} The slope of each crossbeam represents the average rate of change between two points on the graph.

\textbf{Part D:} The slopes of the crossbeams represent the average rate of change, or the slope of the secant line, between two points on the curve.

\textbf{Part E:} The slopes on one side of the parabola are symmetric to the slopes on the other side. So, to find the slopes on one side, we can simply take the negative of the slopes on the other side.

\item Question 12:
To determine the average rate of change in this situation, we first observe that the graph appears to be a parabola with an x-intercept (or T-intercept in this case) of 0 and another value. The average rate of change represents the change in height over the change in time. It tells us how much distance is covered per second on average.

Let's calculate the average rate of change at different given times. We'll calculate it for \(t = 1\) to \(t = 1.5\), and then skip to \(t = 6\). The method is the same for all intervals, but it requires punching numbers into the calculator.

For \(t = 1.5\) to \(t = 1\), the average rate of change is \(\frac{H(1.5) - H(1)}{0.5}\). For \(t = 1\) to \(t = 0.0001\), which is an estimation of an instantaneous rate of change, it's \(\frac{H(1) - H(0.0001)}{0.0001}\).

Upon calculation, we find the following values:

\begin{align*}
\text{From } t &= 1 \text{ to } t = 2: & \text{Average rate of change } &= 5.3 \text{ m/s} \\
\text{From } t &= 1 \text{ to } t = 1.5: & \text{Average rate of change } &= 7.75 \text{ m/s} \\
\text{At } t &= 0.0001: & \text{Instantaneous rate of change } &= 10.2 \text{ m/s}
\end{align*}


Visually, the graph's x-intercepts can be estimated to be around 1 and 4. The slope at \(t = 1\) appears to be about 10, while the average slope from 1 to 2 is 5.3 and from 1 to 1.5 is 7.75. These values can be used to estimate the instantaneous rate of change at \(t = 1\) second.

As we calculate the average rate of change for smaller intervals, such as from \(t = 1\) to \(t = 1.1\), we approach the instantaneous rate of change at \(t = 1\) second, which is approximately 10.2 m/s.

\end{itemize}

\end{document}