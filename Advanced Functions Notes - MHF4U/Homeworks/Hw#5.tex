\documentclass{article}
\usepackage[utf8]{inputenc}
\usepackage[T1]{fontenc}
\usepackage{graphicx}
\usepackage{amsmath, amssymb}
\usepackage{xcolor}
\usepackage{enumerate}
\usepackage{tikz}
\usepackage{lipsum}
\usepackage{sectsty}
\usepackage{polynom}

\title{Practice Questions}
\author{Kensukeken}
\date{March 5th, 2024}
\begin{document}
\maketitle
\section*{Chapter 2.1}
Page 91 \#1-6
\begin{itemize}
\item Question 1:
\begin{enumerate}[a)]
    \item \(\polylongdiv{x^3 + 3x^2 - 2x + 5}{x + 1}\)
    
    \item \begin{align*}
        &\frac{x^3 + 3x^2 - 2x + 5}{x + 1}\\
        &=x^2+2x-4+\frac{9}{x + 1}\\
        &\boxed{x \neq -1}
    \end{align*}
    
    \item \(\mathbb{P}(x)=\mathbb{D}(x)\mathbb{Q}(x)+\mathbb{R}(x)\longrightarrow P(x)=(x+1)(x^2+2x-4)+9\)
    
    \item \begin{align*}
        P(x) &= (x+1)(x^2 + 2x - 4) + 9 \\
        &= x(x^2 + 2x - 4) + 1(x^2 + 2x - 4) + 9 \\
        &= x^3 + 2x^2 - 4x + x^2 + 2x - 4 + 9 \\
        &= x^3 + (2x^2 + x^2) + (2x + 2x) - 4 - 4 + 9 \\
        &= x^3 + 3x^2 + 4x + 5
    \end{align*}
\end{enumerate}
\newpage
\item Question 2:
\begin{enumerate}[a)]
\item \polylongdiv{3x^4 - 4x^3 -6x^2 + 17x - 8}{3x - 4}
\item \begin{align*}
    &x^3-2x+3+\frac{4}{3x-4}\\
    &\boxed{x \neq \frac{4}{3}}
\end{align*}
\item $\mathbb{P}(x)=\mathbb{D}(x)\mathbb{Q}(x)+\mathbb{R}(x) \longrightarrow (3x-4)(x^3-2x+3)+4$
\item \begin{align*}
(3x-4)(x^3-2x+3)+4 &= 3x \cdot x^3 - 3x \cdot 2x + 3x \cdot 3 - 4 \cdot x^3 + 4 \cdot 2x - 4 \cdot 3 + 4 \\
&= 3x^4 - 6x^2 + 9x - 4x^3 + 8x - 12 + 4 \\
&= 3x^4 - 4x^3 - 6x^2 + 17x - 8
\end{align*}
\end{enumerate}
\item Question 3:
\begin{enumerate}[a)]
    \item \(\polylongdiv{x^3 - 7x^2 - 3x + 4}{x - 2}\)
    
    \item \(\polylongdiv{6x^3 - x^2 - 14x - 6}{3x - 2}\)
    
    \item \(\polylongdiv{10x^3 - 11 - 9x^2 - 8x}{5x - 2}\)
    
    \item \(\polylongdiv{11x - 4x^4 - 7}{x - 3}\)
    
    \item \(\polylongdiv{3x^2 - 7x - 6x^3}{3x - 2}\)
    
    \item \(\polylongdiv{8x^3 - 4x^2 - 31}{2x - 3}\)
    
    \item \(\polylongdiv{6x^2 - 6 - 8x^3}{4x - 3}\)
\end{enumerate}
\newpage
\item Question 4:
\begin{enumerate}[a)]
    \item \textbf{Determine the remainder \( R \) so that each statement is true:}
    
    Divide \((2x + 3)(3x - 4)\) and find the remainder \( R \) such that \( R = 6x^2 - x - 15 \).
    
    \[
    \polylongdiv{6x^2 - x - 15}{2x + 3}
    \]

    \item Divide \((x - 2)(x^2 - 3x - 4)\) and find the remainder \( R \) such that \( R = x^3 - x^2 - 2x - 1 \).
    
    \[
    \polylongdiv{x^3 - x^2 - 2x - 1}{x - 2}
    \]

    \item Divide \((x - 4)(2x^2 - 3x - 1)\) and find the remainder \( R \) such that \( R = 2x^3 - 5x^2 - 13x - 2 \).
    
    \[
    \polylongdiv{2x^3 - 5x^2 - 13x - 2}{x - 4}
    \]
\end{enumerate}

\item Question 5:
Given the polynomial expression \(2x^3 - 17x^2 + 38x - 15\), we find that one of the roots is \(x = 5\) using numerical methods or technology such as synthetic division or polynomial long division.

Now, we divide \(2x^3 - 17x^2 + 38x - 15\) by \(x - 5\) to find the other roots.

\[
\polylongdiv{2x^3 - 17x^2 + 38x - 15}{x - 5}
\]

From the division, we get \(2x^3 - 17x^2 + 38x - 15 = (x - 5)(2x^2 - 7x + 3)\).

Now, we solve \(2x^2 - 7x + 3 = 0\) to find the other roots.

The solutions to the quadratic equation \(2x^2 - 7x + 3 = 0\) are \(x = \frac{1}{2}\) and \(x = 3\).

Therefore, the possible dimensions of the box are:

Length = \(x - 5 = \frac{1}{2} - 5 = -\frac{9}{2}\) cm (Discarded since length cannot be negative)

Width = \(x - 5 = 3 - 5 = -2\) cm (Discarded since width cannot be negative)

Height = \(x = 5\) cm

\item Question 6:
Given the polynomial expression for the volume of the box: \(9x^3 - 24x^2 + 44x - 16\), and the polynomial expression for the area of the base: \(9x^2 - 12x + 4\), we need to determine the possible dimensions of the box.

1. Factor the polynomial representing the volume to find its roots:
\[
\polylongdiv{9x^3 - 24x^2 + 44x - 16}{9x^2 - 12x + 4}
\]
From the division, we find that one of the roots is \(x = 2\). Therefore, the height of the box is \(2\) cm.

2. Factor the polynomial representing the area of the base to find its roots:
\[
9x^2 - 12x + 4 = (3x - 2)^2
\]
From this, we find that the side length of the square base is \(3x - 2\).

3. Therefore, the possible dimensions of the square-based box are:
   - Length = Width = \(3x - 2\) cm
   - Height = \(2\) cm

\end{itemize}
\end{document}

