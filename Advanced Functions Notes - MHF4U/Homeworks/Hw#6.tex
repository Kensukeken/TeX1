\documentclass{article}
\usepackage[utf8]{inputenc}
\usepackage[T1]{fontenc}
\usepackage{graphicx}
\usepackage{amsmath, amssymb}
\usepackage{xcolor}
\usepackage{tikz}
\usepackage{lipsum}
\usepackage{sectsty}
\usepackage{polynom}
\usepackage{enumitem}

\title{Practice Questions}
\author{Kensukeken}
\date{March, 7th, 2024}
\begin{document}
\maketitle
\section*{Chapter 2.2 \& 2.3}
\subsection*{Page 91 \#(7-9)eop, 10, 13, 15, 16, 17, 19, 20, 21}
\begin{itemize}
\item Question 7:

\textbf{Use the remainder theorem to determine the remainder when \(2x^3 + 7x^2 - 8x + 3\) is divided by each binomial. Verify your answer using long division:}

a) For \(x + 1\):
\begin{enumerate}
    \item Using the remainder theorem:
    \[
    \text{Remainder} = 2(1)^3 + 7(1)^2 - 8(1) + 3 = 2 + 7 - 8 + 3 = 4
    \]
    \item Verify using long division:
    \[
    \polylongdiv{2x^3 + 7x^2 - 8x + 3}{x + 1}
    \]
    Remainder: \(4\)
\end{enumerate}

b) For \(x - 2\):
\begin{enumerate}
    \item Using the remainder theorem:
    \[
    \text{Remainder} = 2(2)^3 + 7(2)^2 - 8(2) + 3 = 16 + 28 - 16 + 3 = 31
    \]
    \item Verify using long division:
    \[
    \polylongdiv{2x^3 + 7x^2 - 8x + 3}{x - 2}
    \]
    Remainder: \(31\)
\end{enumerate}

c) For \(x + 3\):
\begin{enumerate}
    \item Using the remainder theorem:
    \[
    \text{Remainder} = 2(-3)^3 + 7(-3)^2 - 8(-3) + 3 = -54 + 63 + 24 + 3 = 36
    \]
    \item Verify using long division:
    \[
    \polylongdiv{2x^3 + 7x^2 - 8x + 3}{x + 3}
    \]
    Remainder: \(36\)
\end{enumerate}

d) For \(x - 4\):
\begin{enumerate}
    \item Using the remainder theorem:
    \[
    \text{Remainder} = 2(4)^3 + 7(4)^2 - 8(4) + 3 = 128 + 112 - 32 + 3 = 211
    \]
    \item Verify using long division:
    \[
    \polylongdiv{2x^3 + 7x^2 - 8x + 3}{x - 4}
    \]
    Remainder: \(211\)
\end{enumerate}

e) For \(x - 1\):
\begin{enumerate}
    \item Using the remainder theorem:
    \[
    \text{Remainder} = 2(1)^3 + 7(1)^2 - 8(1) + 3 = 2 + 7 - 8 + 3 = 4
    \]
    \item Verify using long division:
    \[
    \polylongdiv{2x^3 + 7x^2 - 8x + 3}{x - 1}
    \]
    Remainder: \(4\)
\end{enumerate}

\item Question 8:
\textbf{Determine the remainder when each polynomial is divided by \(x + 2\):}

a) For \(x^3 + 3x^2 - 5x + 2\):
\begin{enumerate}
    \item Using the remainder theorem:
    \[
    \text{Remainder} = (2)^3 + 3(2)^2 - 5(2) + 2 = 8 + 12 - 10 + 2 = 12
    \]
    \item Verify using long division:
    \[
    \polylongdiv{x^3 + 3x^2 - 5x + 2}{x - 2}
    \]
    Remainder: \(12\)
\end{enumerate}

b) For \(2x^3 - x^2 - 3x + 1\):
\begin{enumerate}
    \item Using the remainder theorem:
    \[
    \text{Remainder} = 2(2)^3 - (2)^2 - 3(2) + 1 = 16 - 4 - 6 + 1 = 7
    \]
    \item Verify using long division:
    \[
    \polylongdiv{2x^3 - x^2 - 3x + 1}{x - 2}
    \]
    Remainder: \(7\)
\end{enumerate}

c) For \(x^4 + x^3 - 5x^2 + 2x - 7\):
\begin{enumerate}
    \item Using the remainder theorem:
    \[
    \text{Remainder} = (2)^4 + (2)^3 - 5(2)^2 + 2(2) - 7 = 16 + 8 - 20 + 4 - 7 = 1
    \]
    \item Verify using long division:
    \[
    \polylongdiv{x^4 + x^3 - 5x^2 + 2x - 7}{x - 2}
    \]
    Remainder: \(1\)
\end{enumerate}

\item Question 9:
\textbf{Use the remainder theorem to determine the remainder for each division:}

a) For \(x^3 + 2x^2 - 3x + 9\) divided by \(x + 3\):
\begin{enumerate}
    \item Using the remainder theorem:
    \[
    \text{Remainder} = -(3)^3 + 2(3)^2 - (-3)(3) + 9 = -27 + 18 + 9 + 9 = 9
    \]
    \item Verify using long division:
    \[
    \polylongdiv{x^3 + 2x^2 - 3x + 9}{x + 3}
    \]
    Remainder: \(9\)
\end{enumerate}

b) For \(2x^3 + 7x^2 - x + 1\) divided by \(x + 2\):
\begin{enumerate}
    \item Using the remainder theorem:
    \[
    \text{Remainder} = 2(-2)^3 + 7(-2)^2 - (-2) + 1 = -16 + 28 + 2 + 1 = 15
    \]
    \item Verify using long division:
    \[
    \polylongdiv{2x^3 + 7x^2 - x + 1}{x + 2}
    \]
    Remainder: \(15\)
\end{enumerate}


c) For \(x^3 + 2x^2 - 3x + 5\) divided by \(x - 3\):
\begin{enumerate}
    \item Using the remainder theorem:
    \[
    \text{Remainder} = (3)^3 + 2(3)^2 - (3)(3) + 5 = 27 + 18 + 9 + 5 = 41
    \]
    \item Verify using long division:
    \[
    \polylongdiv{x^3 + 2x^2 - 3x + 5}{x - 3}
    \]
    Remainder: \(41\)
\end{enumerate}

d) For \(x^4 - 3x^2 - 5x + 2\) divided by \(x - 2\):
\begin{enumerate}
    \item Using the remainder theorem:
    \[
    \text{Remainder} = (2)^4 - 3(2)^2 - (2) + 2 = 16 - 12 + 10 + 2 = 16
    \]
    \item Verify using long division:
    \[
    \polylongdiv{x^4 - 3x^2 - 5x + 2}{x + 2}
    \]
    Remainder: \(16\)
\end{enumerate}
\item Question 10:
a) Determine the value of \(k\) such that when \(P(x) = kx^3 + 5x^2 - 2x + 3\) is divided by \(x + 1\), the remainder is \(7\):

\begin{enumerate}
    \item Using the remainder theorem:
    \[
    \text{Remainder} = P(-1) = k(-1)^3 + 5(-1)^2 - 2(-1) + 3 = -k + 5 + 2 + 3 = -k + 10
    \]
    Given that the remainder is \(7\), we have:
    \[
    -k + 10 = 7 \implies k = 10 - 7 = 3
    \]
    \item Therefore, \(k = 3\).
\end{enumerate}

b) Determine the remainder when \(P(x)\) is divided by \(x - 3\):

\begin{enumerate}
    \item Using the remainder theorem:
    \[
    \text{Remainder} = P(3) = 3(3)^3 + 5(3)^2 - 2(3) + 3 = 81 + 45 - 6 + 3 = 123
    \]
    \item Verify using long division:
    \[
    \polylongdiv{3x^3 + 5x^2 - 2x + 3}{x - 3}
    \]
    Remainder: \(123\)
\end{enumerate}


\item Question 13:
For what value of \(k\) will the polynomial \(f(x) = x^3 + 6x^2 + kx - 4\) have the same remainder when it is divided by \(x - 1\) and by \(x + 2\)?

Let \(r_1\) and \(r_2\) be the remainders when \(f(x)\) is divided by \(x - 1\) and \(x + 2\), respectively.

\[
\begin{aligned}
    r_1 &= f(1) \\
    &= (1)^3 + 6(1)^2 + k(1) - 4 \\
    &= 1 + 6 + k - 4 \\
    &= k + 3
\end{aligned}
\]

\[
\begin{aligned}
    r_2 &= f(-2) \\
    &= (-2)^3 + 6(-2)^2 + k(-2) - 4 \\
    &= -8 + 24 - 2k - 4 \\
    &= 12 - 2k
\end{aligned}
\]

For \(r_1 = r_2\):

\[
k + 3 = 12 - 2k
\]

Solving for \(k\):

\[
\begin{aligned}
    k + 3 &= 12 - 2k \\
    3k &= 9 \\
    k &= 3
\end{aligned}
\]

Therefore, \(k = 3\) for the polynomial \(f(x)\) to have the same remainder when divided by \(x - 1\) and \(x + 2\).



\item Question 15:

a) Use the remainder theorem to determine the remainder when \(10x^4 - 11x^3 - 8x^2 + 7x + 9\) is divided by \(2x - 3\):

\begin{enumerate}
    \item Using the remainder theorem:
    \[
    \text{Remainder} = f\left(\frac{3}{2}\right) = 10\left(\frac{3}{2}\right)^4 - 11\left(\frac{3}{2}\right)^3 - 8\left(\frac{3}{2}\right)^2 + 7\left(\frac{3}{2}\right) + 9
    \]
    \[
    = 10\left(\frac{81}{16}\right) - 11\left(\frac{27}{8}\right) - 8\left(\frac{9}{4}\right) + 7\left(\frac{3}{2}\right) + 9
    \]
    \[
    = \frac{405}{8} - \frac{297}{8} - 18 + \frac{21}{2} + 9
    \]
    \[
    = \frac{405 - 297 - 144 + 84 + 72}{8}
    \]
    \[
    = \frac{120}{8} = 15
    \]
\end{enumerate}

b) Use technology to verify your answer in part a) using long division or a CAS.
\(\polylongdiv{10x^4 - 11x^3 - 8x^2 + 7x + 9}{2x-3}\)
\item Question 16:

a) Determine the remainder when \(6x^3 + 23x^2 - 6x - 8\) is divided by \(3x - 2\):

\begin{enumerate}
    \item Using the remainder theorem:
    \[
    \text{Remainder} = f\left(\frac{2}{3}\right) = 6\left(\frac{2}{3}\right)^3 + 23\left(\frac{2}{3}\right)^2 - 6\left(\frac{2}{3}\right) - 8
    \]
    \[
    = 6\left(\frac{8}{27}\right) + 23\left(\frac{4}{9}\right) - 6\left(\frac{2}{3}\right) - 8
    \]
    \[
    = \frac{48}{27} + \frac{92}{9} - \frac{12}{3} - 8
    \]
    \[
    = \frac{48}{27} + \frac{276}{27} - \frac{108}{27} - \frac{216}{27}
    \]
    \[
    = \frac{48 + 276 - 108 - 216}{27} = 0
    \]
\end{enumerate}

b) The remainder provides information about \(3x - 2\) because it evaluates \(f(x)\) at the root of the divisor, \(x = \frac{2}{3}\), and yields the leftover value after division, which helps understand how \(f(x)\) behaves in relation to \(3x - 2\).

c) Express \(6x^3 - 23x^2 + 6x - 8\) in factored form.
\begin{align*}
6x^3 - 23x^2 + 6x - 8 &= (6x^3 - 23x^2) + (6x - 8) \\
&= x^2(6x - 23) + 2(3x - 4)
\end{align*}



\item Question 17:

\textbf{Chapter Problem:}
The packaging design team at Best of U has determined that a cost-efficient way of manufacturing cylindrical containers for their products is to have the volume, \(V\), in cubic centimetres, modelled by:
\[ V(x) = 9\pi x^3 - 51\pi x^2 + 88\pi x + 48\pi \]
where \(x\) is an integer such that \(2 \leq x \leq 8\). The height, \(h\), in centimetres, of each cylinder is a linear function given by:
\[ h(x) = x + 3 \]

a) Determine the quotient \(\frac{V(x)}{h(x)}\). Interpret this result.

\[
\frac{V(x)}{h(x)} = \frac{9\pi x^3 - 51\pi x^2 + 88\pi x + 48\pi}{x + 3}
\]

We can simplify this expression by dividing each term in \(V(x)\) by \(h(x)\):
\[
\frac{V(x)}{h(x)} = \frac{9\pi x^3}{x + 3} - \frac{51\pi x^2}{x + 3} + \frac{88\pi x}{x + 3} + \frac{48\pi}{x + 3}
\]

This represents the quotient of \(V(x)\) and \(h(x)\). 

b) Use your answer in part a) to express the volume of a container in the form \(\pi r^2h\).

To express the volume in the form \(\pi r^2h\), we can interpret \(x\) as the radius of the cylinder. Then, the volume formula becomes:
\[
V(x) = \pi r^2 h = \pi (x)^2 (x + 3)
\]

c) What are the possible dimensions and volumes of the containers for the given values of \(x\)?

The possible dimensions of the containers are determined by the range \(2 \leq x \leq 8\). By substituting the values of \(x\) within this range into the expressions for \(V(x)\) and \(h(x)\), we can find the corresponding volumes and heights of the containers.



\item Question 19:

\textbf{a) Determine the remainder:}

Given:
\[ h(t) = -5t^2 + 8.3t + 1.2 \]

We want to find the remainder when \( h(t) \) is divided by \( t - 1.5 \). Using the remainder theorem, we evaluate \( h(1.5) \):
\[ h(1.5) = -5(1.5)^2 + 8.3(1.5) + 1.2 = 2.4 \]

The remainder when \( h(t) \) is divided by \( t - 1.5 \) is \( 2.4 \).

\textbf{b) Interpretation:}

The remainder obtained, \( 2.4 \), represents the height of the shot-put at \( t = 1.5 \) seconds when divided by \( t - 1.5 \). It indicates that the shot-put is at a height of \( 2.4 \) meters at \( t = 1.5 \) seconds.


    \item Question 20:
    Given that the polynomial $mx^3 - 3x^2 + nx + 2$ when divided by $x + 3$ yields a remainder of -1, and when divided by $x - 2$ yields a remainder of -4.
    
    Let's use the remainder theorem to find the values of $m$ and $n$.
    
    \textbf{Using $x + 3$:}
    \[ -27m - 3n = 24 \]
    \[ \Rightarrow -9m - n = 8 \]
    
    \textbf{Using $x - 2$:}
    \[ 8m + 2n = -4 \]
    \[ \Rightarrow 4m + n = -2 \]
    
    Now, we have a system of linear equations:
    \[ 
    \begin{cases}
    -9m - n = 8 \\
    4m + n = -2
    \end{cases}
    \]
    
    Adding these equations eliminates $n$:
    \[ -5m = 6 \]
    \[ m = -\frac{6}{5} \]
    
    Substitute $m = -\frac{6}{5}$ into Equation 2:
    \[ 4\left(-\frac{6}{5}\right) + n = -2 \]
    \[ -\frac{24}{5} + n = -2 \]
    \[ n = -2 + \frac{24}{5} = -2 + \frac{24}{5} \]
    \[ n = -2 + \frac{24}{5} = -\frac{10}{5} + \frac{24}{5} = \frac{14}{5} \]
    
    $\therefore$, $m = -\frac{6}{5}$ and $n = \frac{14}{5}$.
    \newpage 
    \item Question 21:
    Given that the polynomial $3x^3 + ax^2 + bx - 9$ when divided by $x - 2$ yields a remainder of -5, and when divided by $x + 1$ yields a remainder of -16.
    
    Let's use the remainder theorem to find the values of $a$ and $b$.
    
    \textbf{Using $x - 2$:}
    \[ 4a + 2b = -15 \]
    \[ \Rightarrow 2a + b = -\frac{15}{2} \]
    
    \textbf{Using $x + 1$:}
    \[ a - b = -4 \]
    
    Now, we have a system of linear equations:
    \[ 
    \begin{cases}
    2a + b = -\frac{15}{2} \\
    a - b = -4
    \end{cases}
    \]
    
    Adding these equations eliminates $b$:
    \[ 3a = -\frac{23}{2} \]
    \[ a = -\frac{23}{6} \]
    
    Substitute $a = -\frac{23}{6}$ into Equation 2:
    \[ -\frac{23}{6} - b = -4 \]
    \[ - b = -4 + \frac{23}{6} = -4 + \frac{23}{6} \]
    \[ - b = -\frac{24}{6} + \frac{23}{6} = \frac{-24 + 23}{6} \]
    \[ - b = -\frac{1}{6} \]
    \[ b = \frac{1}{6} \]
    
    $\therefore$, $a = -\frac{23}{6}$ and $b = \frac{1}{6}$.

\end{itemize}


\subsection*{Page 102 \#1-11 (eop when appropriate)}
\begin{itemize}
\item Question 1: 
Write the binomial factor that corresponds to the polynomial $P(x)$.

a) $P(4)=0$
\[
\begin{array}{c|cccc}
4 & 1 & 0 & 0 & 0 \\
\hline
 & 1 & 4 & 16 & 64 \\
\end{array}
\]
Since the remainder is not 0, $x - 4$ is not a factor of $P(x)$.

b) $P(-3)=0$
\[
\begin{array}{c|cccc}
-3 & 1 & 0 & 0 & 0 \\
\hline
 & 1 & -3 & 9 & -27 \\
\end{array}
\]
Since the remainder is not 0, $x + 3$ is not a factor of $P(x)$.

c) $P\left(\frac{2}{3}\right)=0$
\[
\begin{array}{c|cccc}
\frac{2}{3} & 1 & 0 & 0 & 0 \\
\hline
 & 1 & \frac{2}{3} & \frac{4}{9} & \frac{8}{27} \\
\end{array}
\]
Since the remainder is not 0, $x - \frac{2}{3}$ is not a factor of $P(x)$.

d) $P\left(-\frac{1}{4}\right)=0$
\[
\begin{array}{c|cccc}
-\frac{1}{4} & 1 & 0 & 0 & 0 \\
\hline
 & 1 & -\frac{1}{4} & \frac{1}{16} & -\frac{1}{64} \\
\end{array}
\]
Since the remainder is not 0, $x + \frac{1}{4}$ is not a factor of $P(x)$.



\item Question 2:
Determine if $x + 3$ is a factor of each polynomial.

a) $x^3+x^2-x+6$
$$\polyhornerscheme[x=-3]{x^3+x^2-x+6}$$

Since the remainder is 0, $x + 3$ is a factor of $x^3+x^2-x+6$.

b) $2x^3+9x^2+10x+3$
$$\polyhornerscheme[x=-3]{2x^3+9x^2+10x+3}$$

Since the remainder is not 0, $x + 3$ is not a factor of $2x^3+9x^2+10x+3$.

c) $x^3+27$\\
$x^3 + 27$ factors as $(x + 3)(x^2 - 3x + 9)$, hence $x + 3$ is a factor of $x^3 + 27$.



\item Question 3

List the values that could be zeros of each polynomial. Then, factor the polynomial.

a) $x^3+3x^2-6x-8$

To find the possible zeros, we use the Rational Root Theorem. The possible rational roots are of the form $\pm \frac{p}{q}$, where $p$ is a factor of the constant term ($-8$) and $q$ is a factor of the leading coefficient ($1$). So the possible rational roots are $\pm 1, \pm 2, \pm 4, \pm 8$.

By synthetic division, we find that $x = -2$ is a zero. Thus, $(x + 2)$ is a factor of $x^3+3x^2-6x-8$. Then we can divide the polynomial by $(x + 2)$ to get:

\[
(x + 2)(x^2 + x - 4)
\]

b) $x^3+4x^2-15x-18$

Using the Rational Root Theorem, the possible rational roots are $\pm 1, \pm 2, \pm 3, \pm 6, \pm 9, \pm 18$. By synthetic division, we find that $x = -3$ is a zero. Thus, $(x + 3)$ is a factor of $x^3+4x^2-15x-18$. Then we can divide the polynomial by $(x + 3)$ to get:

\[
(x + 3)(x^2 + x - 6)
\]

c) $x^3-3x^2-10x+24$

Using the Rational Root Theorem, the possible rational roots are $\pm 1, \pm 2, \pm 3, \pm 4, \pm 6, \pm 8, \pm 12, \pm 24$. By synthetic division, we find that $x = 2$ is a zero. Thus, $(x - 2)$ is a factor of $x^3-3x^2-10x+24$. Then we can divide the polynomial by $(x - 2)$ to get:

\[
(x - 2)(x^2 - 5x - 12)
\]

\item Question 4:
Factor each polynomial by grouping terms.

a) $x^3+x^2-9x-9$

\[
x^3 + x^2 - 9x - 9 = x^2(x + 1) - 9(x + 1) = (x^2 - 9)(x + 1) = (x - 3)(x + 3)(x + 1)
\]

b) $x^3 - x^2 - 16x + 16$

\[
x^3 - x^2 - 16x + 16 = x^2(x - 1) - 16(x - 1) = (x^2 - 16)(x - 1) = (x - 4)(x + 4)(x - 1)
\]

c) $2x^3 - x^2 - 72x + 36$

\[
2x^3 - x^2 - 72x + 36 = x^2(2x - 1) - 36(2x - 1) = (2x^2 - 36)(2x - 1) = 2(x - 6)(x + 6)(2x - 1)
\]

d) $x^3 - 7x^2 - 4x + 28$

\[
x^3 - 7x^2 - 4x + 28 = x^2(x - 7) - 4(x - 7) = (x^2 - 4)(x - 7) = (x - 2)(x + 2)(x - 7)
\]

e) $3x^3 + 2x^2 - 75x - 50$

\[
3x^3 + 2x^2 - 75x - 50 = x^2(3x + 2) - 25(3x + 2) = (3x^2 - 25)(3x + 2) = (x - 5)(x + 5)(3x + 2)
\]

f) $2x^4 + 3x^3 - 32x^2 - 48x$

\[
2x^4 + 3x^3 - 32x^2 - 48x = x^2(2x^2 + 3x - 32) - 48(x^2 + 1) = (2x^4 + 3x^3 - 32x^2 - 48x)(x^2 - 48)
\]


\item Question 5:
Determine the values that could be zeros of each polynomial. Then, factor the polynomial.

a) $3x^3+x^2-22x-24$

Potential rational roots are $\pm1$, $\pm2$, $\pm3$, $\pm4$, $\pm6$, $\pm8$, $\pm12$, $\pm24$.

b) $2x^3-9x^2+10x-3$

Potential rational roots are $\pm1$, $\pm3$, $\pm\frac{1}{2}$, $\pm\frac{3}{2}$.

c) $6x^3-11x^2-26x+15$

Potential rational roots are $\pm1$, $\pm3$, $\pm5$, $\pm15$.

d) $4x^3+3x^2-4x-3$

Potential rational roots are $\pm1$, $\pm3$.


\item Question 6:
Factor each polynomial.

a) $x^3+2x^2-x-2$

\[
x^3 + 2x^2 - x - 2 = (x + 1)(x^2 + x - 2) = (x + 1)(x - 1)(x + 2)
\]

b) $x^3+4x^2-7x-10$

\[
x^3 + 4x^2 - 7x - 10 = (x - 1)(x^2 + 5x + 10)
\]

c) $x^3-5x^2-4x+20$

\[
x^3 - 5x^2 - 4x + 20 = (x - 2)(x^2 - 3x - 10) = (x - 2)(x - 5)(x + 2)
\]

d) $x^3+5x^2+3x-4$

\[
x^3 + 5x^2 + 3x - 4 = (x + 1)(x^2 + 4x - 4)
\]

e) $x^3-4x^2-11x+30$

\[
x^3 - 4x^2 - 11x + 30 = (x + 2)(x^2 - 6x + 15)
\]

f) $x^4-4x^3-x^2+16x-12$

\[
x^4 - 4x^3 - x^2 + 16x - 12 = (x - 2)(x^3 - x^2 + 8)
\]

g) $x^4-2x^3-13x^2+14x+24$

\[
x^4 - 2x^3 - 13x^2 + 14x + 24 = (x - 2)(x^3 - 13x + 12)
\]
\item Question 7:
Use Technology Factor each polynomial.

a) $8x^3+4x^2-2x-1$

\[
8x^3 + 4x^2 - 2x - 1 = (2x + 1)(4x^2 + 2x - 1)
\]

b) $2x^3+5x^2-x-6$

\[
2x^3 + 5x^2 - x - 6 = (2x - 1)(x + 3)(x + 2)
\]

c) $5x^3+3x^2-12x+4$

\[
5x^3 + 3x^2 - 12x + 4 = (5x - 1)(x + 2)(x - 2)
\]

d) $6x^4+x^3-8x^2-x+2$

\[
6x^4 + x^3 - 8x^2 - x + 2 = (2x - 1)(3x^3 + 2x^2 - 1)
\]

e) $5x^4+x^3-22x^2-4x+8$

\[
5x^4 + x^3 - 22x^2 - 4x + 8 = (x + 1)(5x^3 - 5x^2 - 3)
\]

f) $3x^3+4x^2-35x-12$

\[
3x^3 + 4x^2 - 35x - 12 = (x + 3)(3x^2 - 5)
\]

g) $6x^3-17x^2+11x-2$

\[
6x^3 - 17x^2 + 11x - 2 = (2x - 1)(3x^2 - 5)
\]


\item Question 8:
 An artist creates a carving from a rectangular block of soapstone whose volume, $V$, in cubic meters, can be modeled by $V(x)=6x^3 + 25x^2 + 2x - 8$. Determine possible dimensions of the block, in meters, in terms of binomials of $x$.

The volume of a rectangular block is given by the formula: \[ V = lwh \] where $l$, $w$, and $h$ are the length, width, and height respectively. We can express the volume $V(x)$ in terms of the dimensions $l$, $w$, and $h$ as: \[ V(x) = lwh = (2x + 1)(3x - 2)(x + 4) \]

So, the possible dimensions of the block are $2x + 1$ meters, $3x - 2$ meters, and $x + 4$ meters.
\item Question 9:
Determine the value of $k$ so that $x + 2$ is a factor of $x^3 - 2kx^2 + 6x - 4$.

To have $x + 2$ as a factor, we need to have $(x + 2)$ divide the polynomial without any remainder. Therefore, substituting $x = -2$ into the polynomial should result in zero.

\[
\begin{aligned}
    (-2)^3 - 2k(-2)^2 + 6(-2) - 4 &= 0 \\
    -8 - 4k - 12 - 4 &= 0 \\
    -24 - 4k &= 0 \\
    -4k &= 24 \\
    k &= -6
\end{aligned}
\]

So, $k = -6$.
\item Question 10:
Determine the value of $k$ so that $3x - 2$ is a factor of $3x^3 - 5x^2 + kx + 2$.

Similarly, to have $3x - 2$ as a factor, we need to have $(3x - 2)$ divide the polynomial without any remainder. Therefore, substituting $x = \frac{2}{3}$ into the polynomial should result in zero.

\[
\begin{aligned}
    3\left(\frac{2}{3}\right)^3 - 5\left(\frac{2}{3}\right)^2 + \frac{k}{3} + 2 &= 0 \\
    3\left(\frac{8}{27}\right) - 5\left(\frac{4}{9}\right) + \frac{k}{3} + 2 &= 0 \\
    \frac{8}{9} - \frac{20}{9} + \frac{k}{3} + 2 &= 0 \\
    \frac{-12}{9} + \frac{k}{3} + 2 &= 0 \\
    -\frac{4}{3} + \frac{k}{3} + 2 &= 0 \\
    \frac{k}{3} &= \frac{10}{3} \\
    k &= 10
\end{aligned}
\]

So, $k = 10$.

\item Question 11:
Factor each polynomial.

a) $2x^3 + 5x^2 - x - 6$

\[
2x^3 + 5x^2 - x - 6 = (x - 1)(2x^2 + 7x + 6)
\]

b) $4x^3 - 7x - 3$

\[
4x^3 - 7x - 3 = (x - 1)(4x^2 + 4x + 3)
\]

c) $6x^3 + 5x^2 - 21x + 10$

\[
6x^3 + 5x^2 - 21x + 10 = (2x - 1)(3x^2 + 4x - 10)
\]

d) $4x^3 - 8x^2 + 3x - 6$

\[
4x^3 - 8x^2 + 3x - 6 = (x - 1)(4x^2 - 4x + 6)
\]

e) $2x^3 + x^2 + x - 1$

\[
2x^3 + x^2 + x - 1 = (x - 1)(2x^2 + 3)
\]

f) $x^4 - 15x^2 - 10x + 24$

\[
x^4 - 15x^2 - 10x + 24 = (x - 2)(x + 2)(x^2 - 10)
\]



\end{itemize}
\end{document}

