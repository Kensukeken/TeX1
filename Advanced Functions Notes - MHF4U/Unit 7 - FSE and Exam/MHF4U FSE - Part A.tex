\documentclass{article}
\usepackage[utf8]{inputenc}
\usepackage[T1]{fontenc}
\usepackage{amsmath, amssymb}
\usepackage{xcolor}
\usepackage{graphicx}
\usepackage{lipsum}
\usepackage{polynom}
\usepackage{enumitem}
\usepackage{framed}
\usepackage{hyperref}
\usepackage[most]{tcolorbox}
\usepackage{pgfplots}

% Define colors
\definecolor{examplecolor}{RGB}{92, 184, 92}
\definecolor{lessoncolor}{RGB}{74, 144, 226}
\definecolor{notecolor}{RGB}{255, 179, 102}
% Define colors
\definecolor{solutioncolor}{RGB}{74, 144, 226}

% Define new environment for solutions
\newenvironment{solution}{\color{solutioncolor}}{}
% for small text
\newcommand{\smalltext}[1]{\text{\footnotesize #1}}

\title{Advanced Functions Exam - Part A}
\author{Kensukeken}
\date{May 21st, 2024}

\begin{document}
\maketitle

\subsection*{Part A: Short Answer}
1. List the asymptotes of the following:
\begin{enumerate}
    \item[a)] \( y = \frac{2 + 3x}{6 - x} \)
    \begin{solution}
        \begin{align*}
        \text{V.A: } & \; x-6 =0, x = 6 \smalltext{ (Denominator equals zero)} \\
        \text{H.A: } & \; y = \frac{2}{-6} = -3 \smalltext{ (Degree of numerator equals degree of denominator, ratio of leading coefficients)}
    \end{align*}
\end{solution}
    \item[b)] \( y = \frac{2 + 3x}{x^2 + 5x - 14} \)
    \begin{solution}
        \begin{align*}
        &y = \frac{2 + 3x}{(x + 7)(x - 2)} \smalltext{ (Factor out the denominator)} \\
        \text{V.A: } & \; x = x + 7=0, x= -7, x-2=0, x= 2 \smalltext{ (Denominator equals zero)} \\
        \text{H.A: } & \; y = 0 \smalltext{ (Degree of denominator is greater than numerator)}
    \end{align*}
\end{solution}
    \item[c)] \( y = \frac{x^2 - 5x}{x - 1} \)
        \begin{solution}
    \begin{align*}
        &y = \frac{x(x - 5)}{x - 1} \smalltext{ (Factor out numerator)} \\
        \text{V.A: } & \; x -1 = 0, x=1 \smalltext{ (Denominator equals zero)} \\
        \end{align*}
        $$\polylongdiv{x^2-5x}{x-1}$$
    
        \begin{align*}
        \text{O.A: } & \; y = x-4 \smalltext{ (Long division since numerator degree is higher)}
    \end{align*}
    \end{solution}
\end{enumerate}

2. Solve: \( 5 = \frac{2 + 3x}{6 - x} \)
\begin{solution}
\begin{align*}
    5(6 - x) &= 2 + 3x & \text{Expand using distributive property} \\
    30 - 5x &= 2 + 3x & \text{Perform multiplication} \\
    30 - 2 &= 5x + 3x & \text{Combine like terms} \\
    28 &= 5x + 3x & \text{Subtract 2 from both sides} \\
    28 &= 8x & \text{Combine like terms} \\
    x &= \frac{28}{8} & \text{Divide both sides by 8} \\
    x &= \frac{7}{2} & \text{Simplify the fraction} \\
\end{align*} 
    \end{solution}

3. Describe the function \( y=\frac{x^3-5x}{6x^7-4x^3} \) as even, odd or neither.
    \begin{solution}
    \begin{align*}
    y(-x) &= \frac{(-x)^3-5(-x)}{6(-x)^7-4(-x)^3} \\
          &= \frac{-x^3+5x}{-6x^7+4x^3} \\
          &= -\frac{x^3-5x}{6x^7-4x^3} \\
          &= -y(x) \\
    \text{Therefore, the function is odd.}
\end{align*}
    \end{solution}

4. An odd function has 3 vertical asymptotes, one is \( x=3 \). What are the other two?
\begin{solution}
    \begin{align*}
    \text{For an odd function, the asymptotes should be symmetric about the origin.} \\
    \text{Thus, if \( x=3 \) is a vertical asymptote, then the other two are \( x=-3 \) and \( x=0 \).}
\end{align*}
\end{solution}

5. True or False? An even non-constant function that is continuous at \( x = 0 \) has a local max or min there.
\begin{solution}
    \begin{align*}
    &\text{True.} \smalltext{ By symmetry, an even function continuous at } x = 0 \smalltext{ has a local extremum at } x = 0.
\end{align*}
\end{solution}

6. True or False? A reciprocal function has no roots.
\begin{solution}
    \begin{align*}
    &\text{True.} \smalltext{ A reciprocal function of the form } y = \frac{1}{f(x)} \smalltext{ is undefined where } f(x) = 0.
\end{align*}
\end{solution}
\newpage 
7. Convert to radians
\begin{enumerate}
    \item[a)] \( 225^{\circ} \) exact
    \begin{solution}
    \begin{align*}
        225^{\circ} &= 225 \times \frac{\pi}{180} \\
        &= \frac{225\pi}{180} \\
        &= \frac{5\pi}{4}
    \end{align*}
    \end{solution}
    \item[b)] \( 164^{\circ} \) to 3 decimal places
    \begin{solution}
    \begin{align*}
        164^{\circ} &= 164 \times \frac{\pi}{180} \\
        &= \frac{164\pi}{180} \\
        &= \frac{41\pi}{45} \approx 2.862
    \end{align*}
    \end{solution}
\end{enumerate}

8. Convert to degrees
\begin{enumerate}
    \item[a)] \( \frac{7\pi}{12} \)
    \begin{solution}
    \begin{align*}
        \frac{7\pi}{12} &= \frac{7\pi}{12} \times \frac{180}{\pi} \\
        &= \frac{7 \times 180}{12} \\
        &= 105^{\circ}
    \end{align*}
        \end{solution}

    \item[b)] \( 2.34 \) (to 1 decimal place)
    \begin{solution}
    \begin{align*}
        2.34 &= 2.34 \times \frac{180}{\pi} \\
        &= \frac{2.34 \times 180}{\pi} \\
        &\approx 134.1^{\circ}
    \end{align*}
    \end{solution}

\end{enumerate}

9. Determine the angle \( x \in [0, 360^{\circ}] \) and \( \theta \in [0, 2\pi] \)
\begin{enumerate}
    \item[a)] \( \sin x = -0.5 \)
    \begin{solution}
    \begin{align*}
        x &= 210^{\circ}, 330^{\circ} \\
        \theta &= \frac{7\pi}{6}, \frac{11\pi}{6}
    \end{align*}
    \end{solution}

    \item[b)] \( \cot x = -\sqrt{3} \)
    \begin{solution}
    \begin{align*}
        \cot x &= \frac{\cos x}{\sin x} = -\sqrt{3} \\
        x &= 120^{\circ}, 300^{\circ} \\
        \theta &= \frac{2\pi}{3}, \frac{5\pi}{3}
    \end{align*}
    \end{solution}
    \item[c)] \( \sec x = 2.5 \) (to 1 decimal place)
    \begin{solution}
    \begin{align*}
        \sec x &= \frac{1}{\cos x} = 2.5 \implies \cos x = 0.4 \\
        x &\approx 66.4^{\circ}, 293.6^{\circ} \\
        \theta &\approx 1.16, 5.12
    \end{align*}
    \end{solution}
    \item[d)] \( \cos \theta = \frac{1}{\sqrt{2}} \)
    \begin{solution}
    \begin{align*}
        \theta &= \frac{\pi}{4}, \frac{7\pi}{4}
    \end{align*}
    \end{solution}

    \item[e)] \( \csc \theta = 2 \)
    \begin{solution}
        \begin{align*}
        \csc \theta &= \frac{1}{\sin \theta} \implies \sin \theta = \frac{1}{2} \\
        \theta &= \frac{\pi}{6}, \frac{5\pi}{6}
    \end{align*}
    \end{solution}
    \item[f)] \( \cot \theta = 2.5 \) (to 1 decimal place)
\begin{solution}
    \begin{align*}
        \cot \theta &= \frac{1}{\tan \theta} \implies \tan \theta = \frac{1}{2.5} = 0.4 \\
        \theta &\approx 0.38, 3.52
    \end{align*}
    \end{solution}

\end{enumerate}

10. Solve for the angle \( x \in [0, 360^{\circ}] \) and \( \theta \in [0, 2\pi] \)
\begin{enumerate}
    \item[a)] \( \csc^2 x = 2 \)
    \begin{solution}
        \begin{align*}
            \csc^2 x &= 2 \implies \sin^2 x = \frac{1}{2} \implies \sin x = \pm \frac{1}{\sqrt{2}} \\
            x &= 45^{\circ}, 135^{\circ}, 225^{\circ}, 315^{\circ} \\
            \theta &= \frac{\pi}{4}, \frac{3\pi}{4}, \frac{5\pi}{4}, \frac{7\pi}{4}
        \end{align*}
    \end{solution}
    \item[b)] \( 3\sec^2 \theta - 4 = 0 \)
    \begin{solution}
        \begin{align*}
            3\sec^2 \theta &= 4 \implies \sec^2 \theta = \frac{4}{3} \implies \cos^2 \theta = \frac{3}{4} \\
            \cos \theta &= \pm \frac{\sqrt{3}}{2} \\
            \theta &= \frac{\pi}{6}, \frac{5\pi}{6}, \frac{7\pi}{6}, \frac{11\pi}{6}
        \end{align*}
    \end{solution}
    \item[c)] \( \sin 2x = 0.5 \)
    \begin{solution}
        \begin{align*}
            2x &= 30^{\circ}, 150^{\circ}, 390^{\circ}, 510^{\circ} \\
            x &= 15^{\circ}, 75^{\circ}, 195^{\circ}, 255^{\circ} \\
            2x &= \frac{\pi}{6}, \frac{5\pi}{6}, \frac{13\pi}{6}, \frac{17\pi}{6} \\
            x &= \frac{\pi}{12}, \frac{5\pi}{12}, \frac{13\pi}{12}, \frac{17\pi}{12}
        \end{align*}
    \end{solution}
\end{enumerate}

11. Express as a simple trig function of the angle \( x \).
\begin{enumerate}
    \item[a)] \( \sin \left(\frac{\pi}{2} + x\right) \)
    \begin{solution}
        \begin{align*}
            &= \cos x \smalltext{ (Using co-function identity)}
        \end{align*}
    \end{solution}
    \item[b)] \( \sec(-x) \)
    \begin{solution}
        \begin{align*}
            &= \sec x \smalltext{ (Even function property of secant)}
        \end{align*}
    \end{solution}
    \item[c)] \( \tan \left(x - \frac{3\pi}{2}\right) \)
    \begin{solution}
        \begin{align*}
            &= \cot x \smalltext{ (Using period property of tangent)}
        \end{align*}
    \end{solution}
\end{enumerate}

12. Give the period, amplitude, phase shift and axis of \( y = 5\sin(3x - \pi) - 7 \)
\begin{solution}
\begin{align*}
    &\text{Period: } \frac{2\pi}{3} \smalltext{ (Coefficient of } x \smalltext{)} \\
    &\text{Amplitude: } 5 \smalltext{ (Coefficient of sine)} \\
    &\text{Phase shift: } \frac{\pi}{3} \smalltext{ (Solving } 3x - \pi = 0 \smalltext{)} \\
    &\text{Axis: } y = -7 \smalltext{ (Vertical shift down 7 units)}
\end{align*}
\end{solution}

\newpage
13. Simplify
\begin{enumerate}
    \item[a)] \( \sin 13 \cos 25 + \sin 25 \cos 13 \)
    \begin{solution}
    \begin{align*}
        &= \sin(13 + 25) \smalltext{ (Using sum-to-product identities)} \\
        &= \sin 38
    \end{align*}
    \end{solution}

    \item[b)] \( 2\sin 50 \cos 50 \)
    \begin{solution}
    \begin{align*}
        &= \sin 100 \smalltext{ (Using double angle identity for sine)}
    \end{align*}
    \end{solution}

    \item[c)] \( \cos^2 x - 1 \)
    \begin{solution}
    \begin{align*}
        &= -\sin^2 x \smalltext{ (Using Pythagorean identity)}
    \end{align*}
    \end{solution}

    \item[d)] \( \cos(\alpha + 2b) \cos(\alpha + b) \sin(\alpha + b) \)
\begin{solution}
    \begin{align*}
&= \frac{1}{2}[\cos((\alpha + 2b) - (\alpha + b)) + \cos((\alpha + 2b) + (\alpha + b))] \sin(\alpha + b) & \text{(Product-to-sum identity)} \\
&= \frac{1}{2}[\cos(b) + \cos(2\alpha + 3b)] \sin(\alpha + b) & \text{(Simplify the cosine arguments)} \\
&= \frac{1}{2}\cos(b)\sin(\alpha + b) + \frac{1}{2}\cos(2\alpha + 3b)\sin(\alpha + b) & \text{(Distribute the sine term)}
\end{align*}

    \end{solution}
\end{enumerate}

14. Give an exact value for \( \csc 15^{\circ} \)
\begin{solution}
\begin{align*}
    \csc 15^{\circ} &= \frac{1}{\sin 15^{\circ}} && \text{Definition of cosecant.} \\
    &= \frac{1}{\sin(45^{\circ} - 30^{\circ})} && \text{Expressing \(15^{\circ}\) as \(45^{\circ} - 30^{\circ}\).} \\
    &= \frac{1}{\sin 45^{\circ} \cos 30^{\circ} - \cos 45^{\circ} \sin 30^{\circ}} && \text{Applying the subtraction formula for sine.} \\
    &= \frac{1}{\frac{\sqrt{2}}{2} \cdot \frac{\sqrt{3}}{2} - \frac{\sqrt{2}}{2} \cdot \frac{1}{2}} && \text{Substituting values for sine and cosine.} \\
    &= \frac{1}{\frac{\sqrt{6} - \sqrt{2}}{4}} && \text{Simplifying the expression in the denominator.} \\
    &= \frac{4}{\sqrt{6} - \sqrt{2}} \cdot \frac{\sqrt{6} + \sqrt{2}}{\sqrt{6} + \sqrt{2}} && \text{Rationalizing the denominator.} \\
    &= \frac{4(\sqrt{6} + \sqrt{2})}{6 - 2} && \text{Simplifying after multiplication.} \\
    &= \sqrt{6} + \sqrt{2} && \text{Final simplification.}
\end{align*}
\end{solution}
\newpage
15. The population of trout in a river is given by \( N(t) = 1000 + \frac{1000t^2}{t^2 + 100} \), \( t \geq 0 \).
\begin{enumerate}
    \item[a)] What size will the trout population be after a long time?
    \begin{solution}

\begin{align*}
& \text{ Initial Expression} \\
& \lim_{t \rightarrow \infty} N(t) = 1000 + \frac{1000}{1 + \lim_{t \rightarrow \infty} \left[\frac{100}{t^2}\right]} \quad \text{(Evaluate the Inner Limit)}  \\
& \text{As } t \text{ approaches infinity, the term } \frac{100}{t^2} \text{ approaches 0.} \\
& \lim_{t \rightarrow \infty} \left[\frac{100}{t^2}\right] = 0 \\
& \text{Substitute the Inner Limit Result} \\
& \text{Substitute 0 for } \lim_{t \rightarrow \infty} \left[\frac{100}{t^2}\right] \text{ in the initial expression.} \\
& \lim_{t \rightarrow \infty} N(t) = 1000 + \frac{1000}{1 + 0} = 1000 + 1000 \\
& \text{ Final Result:} \\
& \text{The limit of } N(t) \text{ as } t \text{ approaches infinity is 2000 Trout.} \\
& \lim_{t \rightarrow \infty} N(t) = 2000 \text{ Trout}
\end{align*}



    \end{solution}

    \item[b)] How many trout were in the river to begin with?
\begin{solution}
    \begin{align*}
        N(0) &= 1000 + \frac{1000 \cdot 0^2}{0^2 + 100} \\
             &= 1000 + 0 \\
             &= 1000\\\\
        &\text{Therefore, the trout were in the river to begin with 1000  }     
    \end{align*}
\end{solution}
    \item[c)] How fast is the trout population growing at three years?
    \begin{solution}
    \begin{align*}
       N(3)=1000+\frac{1000 \cdot 3}{3^2+100}=1000+\frac{9000}{109} \approx 1082.569
    \end{align*}
\end{solution}

    \item[d)] What is the average population growth for the first three years?
   \begin{solution}
    \begin{align*}
        \text{Average growth rate} &= \frac{N(3) - N(0)}{3 - 0} \\
        N(3) &= 1000 + \frac{1000 \cdot 3^2}{3^2 + 100} \\
             &= 1000 + \frac{9000}{109} \\
             &= 1000 + 82.57 \\
             &= 1082.57 \\
        \text{Average growth rate} &= \frac{1082.57 - 1000}{3} \\
                                   &= \frac{82.57}{3} \\
                                   &\approx 27.52 \smalltext{ trout per year}
    \end{align*}
\end{solution} 

\end{enumerate}

16. The probability, P of hitting a target x feet away is graphed below.
\begin{enumerate}
    \item[a)] What is the rate of change of P as player moves 10 ft to 90 ft away?
    \item[b)] How fast is the probability changing when the participant is 50 ft away?
\end{enumerate}
\begin{center}
\begin{tikzpicture}[scale=0.05]
    % Draw axes
    \draw[->] (-25,0) -- (125,0) node[right] {$x$};
    \draw[->] (0,-10) -- (0,110) node[above] {$P(x)$};
    
    % Draw ticks on x-axis
    \foreach \x in {-20, -10,  10, 20, 40, 60, 80, 100, 120}
        \draw (\x,2) -- (\x,-2) node[below] {\tiny \x};
        
    % Draw ticks on y-axis
    \foreach \y in {20, 40,50,60,70,80,90,100}
        \draw (2,\y) -- (-2,\y) node[left] {\tiny \y};
    
    % Plot points
    \foreach \Point in {(-20,55), (-10,70), (0,100), (10,95), (20,80), (40,35), (60,25), (100,15), (120,10)}
        \node at \Point {\textbullet};
            \draw[dashed] (-20,55) -- (-10,70) -- (0,100) -- (10,95) -- (20,80) -- (40,35) -- (60,25) -- (100,15) -- (120,10);
\end{tikzpicture}
\end{center}
\newpage 
\begin{solution}
    \begin{enumerate}
    \item[a)] The average rate of change of \( P \) from 10 ft to 90 ft is calculated as follows:
    \[
    \text{Average rate of change} = \frac{P(90) - P(10)}{90 - 10}
    \]
 After plugging in the values from the graph, we get:
    \[
    \text{Average rate of change} = \frac{80 - 30}{90 - 10} = \frac{50}{80} = 0.625
    \]
    So, the average rate of change is \( 0.625 \) approximately.

    
    \item[b)] The instantaneous rate of change of \( P \) at 50 ft is the derivative of \( P \) at that point:
    \[
    P'(50) = \lim_{h \to 0} \frac{P(50+h) - P(50)}{h}
    \]

\begin{align*}
P'(50) &= \lim_{h \to 0} \frac{P(50+h) - P(50)}{h} \\
&= \lim_{h \to 0} \frac{P(50) + P'(50)h - P(50)}{h} \quad \text{(Using the definition of the derivative)} \\
&= \lim_{h \to 0} \frac{P'(50)h}{h} \quad \text{(Canceling out \(P(50)\))} \\
&= \lim_{h \to 0} P'(50) \\
&= P'(50)
\end{align*}
Or
\[
P'(50) = \frac{d}{dx} P(x) \bigg|_{x=50}
\]

In simpler terms, we're figuring out how fast the probability changes at 50 ft by looking at the slope of the curve at that point. We do this by drawing a line that just touches the curve at 50 ft and finding its steepness. The result is \( P'(50) \), which tells us the rate of change at that exact spot.
    
\end{enumerate}

\begin{align*}
    \text{\tiny Slope} &= \frac{{\text{\tiny Change in } y}}{{\text{\tiny Change in } x}} \\
    &= \frac{{-10}}{{20}} \\
    &= -\frac{1}{2}
\end{align*}

Using the point-slope form of a line:
\begin{align*}
    y - 35 &= -\frac{1}{2}(x - 40) \\
    y - 35 &= -\frac{1}{2}x + 20 \\
    y &= -\frac{1}{2}x + 55
\end{align*}

So, the equation of the tangent line passing through the points \( (40, 35) \) and \( (60, 25) \) is \( y = -\frac{1}{2}x + 55 \).

\end{solution}
\begin{center}
\begin{tikzpicture}[scale=0.05]
    % Draw axes
    \draw[->] (-25,0) -- (125,0) node[right] {$x$};
    \draw[->] (0,-10) -- (0,110) node[above] {$P(x)$};
    
    % Draw ticks on x-axis
    \foreach \x in {-20, -10,  10, 20, 40, 60, 80, 100, 120}
        \draw (\x,2) -- (\x,-2) node[below] {\tiny \x};
        
    % Draw ticks on y-axis
    \foreach \y in {20, 40,50,60,70,80,90,100}
        \draw (2,\y) -- (-2,\y) node[left] {\tiny \y};
    
    % Plot points
    \foreach \Point in {(-20,55), (-10,70), (0,100), (10,95), (20,80), (40,35), (60,25), (100,15), (120,10)}
        \node at \Point {\textbullet};
            \draw[dashed] (-20,55) -- (-10,70) -- (0,100) -- (10,95) -- (20,80) -- (40,35) -- (60,25) -- (100,15) -- (120,10);
\draw[red] (20,45) -- (100,5) node at (80, 50)[left] {$y = -\frac{1}{2}x + 55$};
    
\end{tikzpicture}
\end{center}
\end{document}