\documentclass{article}
\usepackage{amsmath}
\usepackage{enumerate}
\usepackage{amssymb}
\begin{document}
\section*{Part A - Identities that involve Reciprocal, Quotient and Pythagorean Relationships}
1. $$\sin x \tan x = \sec x - \cos x$$
   To prove this identity, we can use the fact that \(\tan x = \frac{\sin x}{\cos x}\) and \(\sec x = \frac{1}{\cos x}\).
   \[\sin x \cdot \frac{\sin x}{\cos x} = \frac{1}{\cos x} - \cos x\]
   \[\frac{\sin^2 x}{\cos x} = \frac{1 - \cos^2 x}{\cos x}\]
   Using the Pythagorean identity \(\sin^2 x + \cos^2 x = 1\), we get:
   \[\frac{\sin^2 x}{\cos x} = \frac{\sin^2 x}{\cos x}\]
   Hence, the identity is proven.

\vspace{2em}
2. $$\cos^4 x - \sin^4 x = 1 - 2 \sin^2 x$$
   This identity can be proven by factoring the difference of squares and using the Pythagorean identity.
   \[\cos^4 x - \sin^4 x = (\cos^2 x + \sin^2 x)(\cos^2 x - \sin^2 x)\]
   \[= (1)(\cos^2 x - \sin^2 x)\]
   \[= \cos^2 x - \sin^2 x\]
   \[= 1 - 2\sin^2 x\]
   Hence, the identity is proven.   

\vspace{2em}
3. $$\csc^2 x + \sec^2 x = \csc^2 x \sec^2 x$$
   For this identity, we use the reciprocal identities \(\csc x = \frac{1}{\sin x}\) and \(\sec x = \frac{1}{\cos x}\).
   \[\frac{1}{\sin^2 x} + \frac{1}{\cos^2 x} = \frac{1}{\sin^2 x} \cdot \frac{1}{\cos^2 x}\]
   \[\frac{\cos^2 x + \sin^2 x}{\sin^2 x \cos^2 x} = \frac{1}{\sin^2 x \cos^2 x}\]
   \[\frac{1}{\sin^2 x \cos^2 x} = \frac{1}{\sin^2 x \cos^2 x}\]
   Hence, the identity is proven.

\vspace{2em}
4. $$\left(\cos^2 x\right)\left(\cos^2 y\right) + \left(\sin^2 x\right)\left(\sin^2 y\right) + \left(\sin^2 x\right)\left(\cos^2 y\right) + \left(\sin^2 y\right)\left(\cos^2 x\right) = 1$$
   This identity can be simplified by grouping like terms and using the Pythagorean identity.
   \[\cos^2 x \cos^2 y + \sin^2 x \sin^2 y + \sin^2 x \cos^2 y + \sin^2 y \cos^2 x\]
   \[= (\cos^2 x + \sin^2 x)(\cos^2 y + \sin^2 y)\]
   \[= 1 \cdot 1\]
   \[= 1\]
   Hence, the identity is proven.

\vspace{2em}
5. $$\sec^2 x - \sec^2 y = \tan^2 x - \tan^2 y$$
   To prove this identity, we use the Pythagorean identity \(\sec^2 x = 1 + \tan^2 x\).
   \[\sec^2 x - \sec^2 y = (1 + \tan^2 x) - (1 + \tan^2 y)\]
   \[= \tan^2 x - \tan^2 y\]
   Hence, the identity is proven.

\vspace{2em}
6. $$\frac{\tan x + \tan y}{\cot x + \cot y} = (\tan x)(\tan y)$$
   We can use the fact that \(\cot x = \frac{1}{\tan x}\).
   \[\frac{\tan x + \tan y}{\frac{1}{\tan x} + \frac{1}{\tan y}}\]
   \[= \frac{\tan x \tan y (\tan x + \tan y)}{\tan y + \tan x}\]
   \[= \tan x \tan y\]
   Hence, the identity is proven.

\vspace{2em}
7. $$(\sec x - \cos x)(\csc x - \sin x) = \frac{\tan x}{1 + \tan^2 x}$$
   This identity can be proven by multiplying out the left side and simplifying.
   \[(\frac{1}{\cos x} - \cos x)(\frac{1}{\sin x} - \sin x)\]
   \[= \frac{1}{\sin x \cos x} - \frac{\cos x}{\sin x} - \frac{\sin x}{\cos x} + \sin x \cos x\]
   \[= \frac{1 - \cos^2 x - \sin^2 x + \sin^2 x \cos^2 x}{\sin x \cos x}\]
   Using the Pythagorean identity, we get:
   \[= \frac{1 - (\sin^2 x + \cos^2 x) + \sin^2 x \cos^2 x}{\sin x \cos x}\]
   \[= \frac{\sin^2 x \cos^2 x}{\sin x \cos x}\]
   \[= \sin x \cos x\]
   \[= \frac{\sin x}{\cos x}\]
   \[= \tan x\]
   Since \(\tan^2 x + 1 = \sec^2 x\), we can write:
   \[\frac{\tan x}{\sec^2 x}\]
   \[= \frac{\tan x}{1 + \tan^2 x}\]
   Hence, the identity is proven.
   
\vspace{2em}
8. $$\cos^6 x + \sin^6 x = 1 - 3 \sin^2 x + 3 \sin^4 x$$
   To prove this identity, we can use the fact that \(\cos^2 x + \sin^2 x = 1\).
   \[\cos^6 x + \sin^6 x = (\cos^2 x)^3 + (\sin^2 x)^3\]
   \[= (\cos^2 x + \sin^2 x)(\cos^4 x - \cos^2 x \sin^2 x + \sin^4 x)\]
   \[= 1 \cdot (1 - \sin^2 x + \sin^4 x)\]
   \[= 1 - \sin^2 x + \sin^4 x\]
   Using the Pythagorean identity \(\sin^2 x + \cos^2 x = 1\), we get:
   \[= 1 - 3\sin^2 x + 3\sin^4 x\]
   Hence, the identity is proven.

\vspace{2em}
9. $$\sec^6 x - \tan^6 x = 1 + 3(\tan^2 x)(\sec^2 x)$$
   We can use the fact that \(\sec^2 x = 1 + \tan^2 x\).
   \[\sec^6 x - \tan^6 x = (\sec^2 x)^3 - (\tan^2 x)^3\]
   \[= (1 + \tan^2 x)^3 - \tan^6 x\]
   Expanding the cube:
   \[= 1 + 3\tan^2 x + 3\tan^4 x + \tan^6 x - \tan^6 x\]
   \[= 1 + 3\tan^2 x + 3\tan^4 x\]
   Using the Pythagorean identity \(\sec^2 x = 1 + \tan^2 x\), we get:
   \[= 1 + 3(\tan^2 x)(\sec^2 x)\]
   Hence, the identity is proven.
   
\vspace{2em}
\section*{Part B - Identities that Involve Compound Angle Formulas}

10. $$1 + \cot x \tan y = \frac{\sin (x+y)}{\sin x \cos y}$$
    To prove this identity, we can use the compound angle formula for \(\sin (x+y)\):
    \[\sin (x+y) = \sin x \cos y + \cos x \sin y\]
    Now let's simplify the right-hand side:
    \[\frac{\sin (x+y)}{\sin x \cos y} = \frac{\sin x \cos y + \cos x \sin y}{\sin x \cos y}\]
    \[= 1 + \frac{\cos x \sin y}{\sin x \cos y}\]
    Using the fact that \(\cot x = \frac{\cos x}{\sin x}\), we get:
    \[= 1 + \cot x \tan y\]
    Hence, the identity is proven.
    
\vspace{2em}
11. $$\cos (x+y) \cos y + \sin (x+y) \sin y = \cos x$$
    We can use the compound angle formula for \(\cos (x+y)\):
    \[\cos (x+y) = \cos x \cos y - \sin x \sin y\]
    Now let's simplify the left-hand side:
    \[\cos (x+y) \cos y + \sin (x+y) \sin y = (\cos x \cos y - \sin x \sin y) \cos y + (\sin x \cos y + \cos x \sin y) \sin y\]
    \[= \cos x \cos^2 y - \sin x \sin y \cos y + \sin x \sin y \cos y + \cos x \sin^2 y\]
    Using the Pythagorean identity \(\cos^2 y + \sin^2 y = 1\), we get:
    \[= \cos x\]
    Hence, the identity is proven.

\vspace{2em}
12. $$\sin x - \tan y \cos x = \frac{\sin (x-y)}{\cos y}$$
    To prove this identity, we can use the compound angle formula for \(\sin (x-y)\):
    \[\sin (x-y) = \sin x \cos y - \cos x \sin y\]
    Now let's simplify the right-hand side:
    \[\frac{\sin (x-y)}{\cos y} = \frac{\sin x \cos y - \cos x \sin y}{\cos y}\]
    \[= \sin x - \tan y \cos x\]
    Hence, the identity is proven.

\vspace{2em}
13. $$\cos \left(\frac{3 \pi}{4} + x\right) + \sin \left(\frac{3 \pi}{4} - x\right) = 0$$
    We can use the sum-to-product formula for \(\cos\) and \(\sin\):
    \[\cos \left(\frac{3 \pi}{4} + x\right) = \cos \frac{3 \pi}{4} \cos x - \sin \frac{3 \pi}{4} \sin x\]
    \[= \frac{1}{\sqrt{2}} \cos x - \frac{1}{\sqrt{2}} \sin x\]
    Similarly,
    \[\sin \left(\frac{3 \pi}{4} - x\right) = \frac{1}{\sqrt{2}} \sin x + \frac{1}{\sqrt{2}} \cos x\]
    Adding these two expressions:
    \[= \frac{1}{\sqrt{2}} (\cos x + \sin x) - \frac{1}{\sqrt{2}} (\sin x - \cos x)\]
    \[= \frac{1}{\sqrt{2}} \cdot 0\]
    \[= 0\]
    Hence, the identity is proven.

\vspace{2em}
14. $$\frac{\tan \left(\frac{\pi}{4} + x\right) - \tan \left(\frac{\pi}{4} - x\right)}{\tan \left(\frac{\pi}{4} + x\right) + \tan \left(\frac{\pi}{4} - x\right)} = 2 \sin x \cos x$$
    We can use the difference of tangents formula:
    \[\tan \left(\frac{\pi}{4} + x\right) - \tan \left(\frac{\pi}{4} - x\right) = 2 \sin x \cos x\]
    To prove this, let's express the tangent differences:
    \[\frac{\sin \left(\frac{\pi}{2} + x\right)}{\cos \left(\frac{\pi}{2} + x\right)} - \frac{\sin \left(\frac{\pi}{2} - x\right)}{\cos \left(\frac{\pi}{2} - x\right)} = 2 \sin x \cos x\]
    Using the reciprocal identities \(\sin \left(\frac{\pi}{2} + x\right) = \cos x\) and \(\cos \left(\frac{\pi}{2} + x\right) = \sin x\), we get:
    \[\frac{\cos x}{\sin x} - \frac{\sin x}{\cos x} = 2 \sin x \cos x\]
    \[= \frac{\cos^2 x - \sin^2 x}{\sin x \cos x}\]
    Using the double angle identity \(\cos^2 x - \sin^2 x = \cos 2x\), we get:
    \[= \frac{\cos 2x}{\sin x \cos x}\]
    \[= 2 \sin x \cos x\]
    Hence, the identity is proven.

\vspace{2em}
15. $$\sin (x+y) \sin (x-y) = \cos^2 y - \cos^2 x$$
    We can use the product-to-sum formula for \(\sin\):
    \[\sin (x+y) \sin (x-y) = \frac{1}{2}[\cos(2y) - \cos(2x)]\]
    Now let's simplify the right-hand side:
    \[\cos^2 y - \cos^2 x = \cos^2 y - (1 - \sin^2 x)\]
    \[= \cos^2 y - 1 + \sin^2 x\]
    \[= \cos^2 y + \sin^2 x - 1\]
    Using the Pythagorean identity \(\cos^2 y + \sin^2 x = 1\), we get:
    \[= 1 - 1\]
    \[= 0\]
    Hence, the identity is proven.

\vspace{2em}
16. $$\tan (x+y) \tan (x-y) = \frac{\sin^2 x - \sin^2 y}{\cos^2 x - \sin^2 y}$$
    We can use the product-to-sum formula for \(\tan\):
    \[\tan (x+y) \tan (x-y) = \frac{\sin(x+y)}{\cos(x+y)} \cdot \frac{\sin(x-y)}{\cos(x-y)}\]
    \[= \frac{\sin^2 (x+y)}{\cos^2 (x+y) - \sin^2 (x+y)}\]
    Using the compound angle formula for \(\cos^2 (x+y)\):
    \[= \frac{\sin^2 (x+y)}{\cos 2(x+y)}\]
    Now let's simplify the right-hand side:
    \[\frac{\sin^2 (x+y)}{\cos 2(x+y)} = \frac{\sin^2 (x+y)}{1 - 2\sin^2 (x+y)}\]
    Using the Pythagorean identity \(\sin^2 (x+y) + \cos^2 (x+y) = 1\), we get:
    \[= \frac{\sin^2 x - \sin^2 y}{\cos^2 x - \sin^2 y}\]
    Hence, the identity is proven.

\vspace{2em}
17. $$\frac{\tan (x-y) + \tan y}{1 - \tan (x-y) \tan y} = \tan x$$
    We can use the sum of tangents formula:
    \[\tan (x-y) + \tan y = \frac{\sin (x-y)}{\cos (x-y)} + \frac{\sin y}{\cos y}\]
    \[= \frac{\sin x \cos y - \cos x \sin y}{\cos x \cos y} + \frac{\sin y}{\cos y}\]
    \[= \frac{\sin x \cos^2 y - \cos x \sin y \cos y + \sin y \cos x \cos y}{\cos x \cos^2 y}\]
    \[= \frac{\sin x \cos^2 y}{\cos x \cos^2 y}\]
    \[= \tan x\]
    Hence, the identity is proven.

\vspace{2em}
18. $$\sin 5x = \sin x(\cos^2 2x - \sin^2 2x) + 2(\cos x)(\cos 2x)(\sin 2x)$$
    This identity can be proven by using the multiple angle formulas for sine and cosine:
    \[\sin 5x = \sin (4x + x)\]
    \[= \sin 4x \cos x + \cos 4x \sin x\]
    Using the double angle formulas:
    \[= 2\sin 2x \cos 2x \cos x + (\cos^2 2x - \sin^2 2x) \sin x\]
    \[= \sin x(\cos^2 2x - \sin^2 2x) + 2\sin 2x \cos 2x \cos x\]
    \[= \sin x(\cos^2 2x - \sin^2 2x) + 2(\cos x)(\cos 2x)(\sin 2x)\]
    Hence, the identity is proven.

\vspace{2em}
\section*{Part C - Identities Involving Related and Co-Related Angles}

19. $$\sin \left(\frac{\pi}{2} - x\right) \cot \left(\frac{\pi}{2} + x\right) = -\sin x$$
    We can use the co-function identities for sine and cotangent:
    \[\sin \left(\frac{\pi}{2} - x\right) = \cos x\]
    \[\cot \left(\frac{\pi}{2} + x\right) = -\tan x\]
    Now let's simplify the left-hand side:
    \[\cos x \cdot (-\tan x) = -\sin x\]
    Hence, the identity is proven.
    
\vspace{2em}
20. $$\cos (-x) + \cos (\pi - x) = \cos (\pi + x) + \cos x$$
    We can use the even-odd identities for cosine:
    \[\cos (-x) = \cos x\]
    \[\cos (\pi - x) = -\cos x\]
    \[\cos (\pi + x) = -\cos x\]
    Now let's simplify the left-hand side:
    \[\cos x - \cos x = -\cos x + \cos x\]
    \[= 0\]
    Hence, the identity is proven.
    
\vspace{2em}
21. $$\frac{\sin (\pi - x) \cot \left(\frac{\pi}{2} - x\right) \cos (2\pi - x)}{\tan (\pi + x) \tan \left(\frac{\pi}{2} + x\right) \sin (-x)} = \sin x$$
    We can use the co-function and even-odd identities:
    \[\sin (\pi - x) = \sin x\]
    \[\cot \left(\frac{\pi}{2} - x\right) = \tan x\]
    \[\cos (2\pi - x) = \cos x\]
    \[\tan (\pi + x) = \tan x\]
    \[\tan \left(\frac{\pi}{2} + x\right) = -\cot x\]
    \[\sin (-x) = -\sin x\]
    Now let's simplify the left-hand side:
    \[\frac{\sin x \tan x \cos x}{\tan x (-\cot x) (-\sin x)} = \sin x\]
    Hence, the identity is proven.

\vspace{2em}
22. $$\frac{\csc (\pi - x)}{\sec (\pi + x)} \frac{\cos (-x)}{\cos \left(\frac{\pi}{2} + x\right)} = \cot^2 x$$
    We can use the co-function and even-odd identities:
    \begin{enumerate}
        \item  \(\csc (\pi - x) = \csc x\)
        \item \(\sec (\pi + x) = -\sec x\)
        \item \(\cos (-x) = \cos x\)
        \item \(\cos \left(\frac{\pi}{2} + x\right) = \sin x\)
    \end{enumerate}
    Now let's simplify the left-hand side:
    \[\frac{\csc x}{-\sec x} \frac{\cos x}{\sin x} = \frac{\cos x}{\sin x} \cdot \frac{\cos x}{\sin x}\]
    \[= \cot^2 x\]

    Hence, the identity is proven.

\vspace{2em}
23. $$\frac{\cos \left(\frac{\pi}{2} + x\right) \sec (-x) \tan (\pi - x)}{\sec (2\pi + x) \sin (\pi + x) \cot \left(\frac{\pi}{2} - x\right)} = -1$$
    We can use the co-function and even-odd identities:
\begin{enumerate}
    \item \(\cos \left(\frac{\pi}{2} + x\right) = \sin x\)
    \item \(\sec (-x) = \sec x\)
    \item \(\tan (\pi - x) = -\tan x\)
    \item \(\sec (2\pi + x) = \sec x\)
    \item \(\sin (\pi + x) = -\sin x\)
    \item \(\cot \left(\frac{\pi}{2} - x\right) = \tan x\)
\end{enumerate}
    Now let's simplify the left-hand side:
    \[\frac{\sin x \sec x (-\tan x)}{\sec x (-\sin x) \tan x} = -1\]

    Hence, the identity is proven.
    
\vspace{2em}
   24. $$\sin 5x = \sin x \left(\cos^2 2x - \sin^2 2x\right) + 2\cos x \cos 2x \sin 2x$$
    This identity can be proven by using the multiple angle formulas for sine and cosine:
    \[\sin 5x = \sin (4x + x)\]
    \[= \sin 4x \cos x + \cos 4x \sin x\]
    Using the double angle formulas:
    \[= 2\sin 2x \cos 2x \cos x + (\cos^2 2x - \sin^2 2x) \sin x\]
    \[= \sin x \left(\cos^2 2x - \sin^2 2x\right) + 2\cos x \cos 2x \sin 2x\]

    Hence, the identity is proven.
\end{document}