\documentclass[12pt,a4paper]{article}
\usepackage{amsmath,amsfonts,amssymb}
\usepackage{graphicx}
\usepackage{enumitem}
\usepackage[dvipsnames]{xcolor}
\usepackage{pgfplots}
\usepackage{hyperref}
\usepackage{soul}
\usepackage{framed}


\definecolor{lessonbgcolor}{rgb}{0.9,0.9,1}
\definecolor{examplecolor}{rgb}{0.8,1,0.8}

\newenvironment{lesson}[1]
  {\begin{framed}\colorbox{lessonbgcolor}{
  \parbox{\dimexpr\linewidth-2\fboxsep}{
  \textbf{#1}}}\end{framed}}
  
\newenvironment{example}
  {\begin{framed}\colorbox{examplecolor}{
  \parbox{\dimexpr\linewidth-2\fboxsep}{
  \textbf{Example:}}}}
  {\end{framed}}


\pgfplotsset{width=7cm,compat=1.17}
\title{Grade 11 Functions Notes: Unit 1}
\author{Made By Kensukeken}
\date{\today}

\begin{document}
\section*{Harcourt TextBook Page 52}
\begin{lesson}{PART A}

\subsection*{1. Factor each of the following:}

\begin{example}
a. $ab - bc$
\[
ab - bc = b(a - c)
\]
\end{example}

\begin{example}
b. $m^4 + m^2$
\[
m^4 + m^2 = m^2(m^2 + 1)
\]
\end{example}

\begin{example}
c. $2x - 8$
\[
2x - 8 = 2(x - 4)
\]
\end{example}

\begin{example}
d. $x^2 + x$
\[
x^2 + x = x(x + 1)
\]
\end{example}

\begin{example}
e. $y^2 - 1$
\[
y^2 - 1 = (y + 1)(y - 1)
\]
\end{example}
\newpage
\subsection*{2. Factor each of the following:}

\begin{example}
a. $4 - a^2$
\[
4 - a^2 = (2 + a)(2 - a)
\]
\end{example}

\begin{example}
b. $25 - t^4$
\[
25 - t^4 = (5 + t^2)(5 - t^2) = (5 + t^2)(5 + t)(5 - t)
\]
\end{example}

\begin{example}
c. $m^2 - 16$
\[
m^2 - 16 = (m + 4)(m - 4)
\]
\end{example}

\begin{example}
d. $x^2y^2 - 9$
\[
x^2y^2 - 9 = (xy + 3)(xy - 3)
\]
\end{example}

\begin{example}
e. $x^2 - 3x + 2$
\[
x^2 - 3x + 2 = (x - 2)(x - 1)
\]
\end{example}
\newpage 
\subsection*{3. Factor each of the following:}

\begin{example}
a. $t^2 + 7t + 12$
\[
t^2 + 7t + 12 = (t + 3)(t + 4)
\]
\end{example}

\begin{example}
b. $x^2 - 5x - 6$
\[
x^2 - 5x - 6 = (x - 6)(x + 1)
\]
\end{example}

\begin{example}
c. $y^2 + 6y + 5$
\[
y^2 + 6y + 5 = (y + 5)(y + 1)
\]
\end{example}

\begin{example}
d. $x^2 - 3x + 2$
\[
x^2 - 3x + 2 = (x - 2)(x - 1)
\]
\end{example}

\begin{example}
e. $ab - bc$
\[
ab - bc = b(a - c)
\]
\end{example}

\end{lesson}
\newpage 
\begin{lesson}{PART B}

\subsection*{4. Factor completely each of the following:}

\begin{example}
a. $x(2x - 1) + 2(2x - 1)$
\[
x(2x - 1) + 2(2x - 1) = (2x - 1)(x + 2)
\]
\end{example}

\begin{example}
b. $9a^3 - 12a$
\[
9a^3 - 12a = 3a(3a^2 - 4)
\]
\end{example}

\begin{example}
c. $4a^3b^4 - 6a^2b^2 + 2ab$
\[
4a^3b^4 - 6a^2b^2 + 2ab = 2ab(2a^2b^3 - 3ab + 1)
\]
\end{example}

\begin{example}
d. $\pi r^2 + \pi rh$
\[
\pi r^2 + \pi rh = \pi r(r + h)
\]
\end{example}

\begin{example}
e. $9x^3 - 6x^2$
\[
9x^3 - 6x^2 = 3x^2(3x - 2)
\]
\end{example}

\begin{example}
f. $(x + 3)^2 - 3(x + 3)$
\[
(x + 3)^2 - 3(x + 3) = (x + 3)(x + 3 - 3) = (x + 3)(x)
\]
\end{example}
\subsection*{5. Factor completely each of the following:}

\begin{example}
a. $25 - x^2y^2$
\[
25 - x^2y^2 = (5 + xy)(5 - xy)
\]
\end{example}

\begin{example}
b. $\pi r^2 - \pi rh$
\[
\pi r^2 - \pi rh = \pi r(r - h)
\]
\end{example}


\begin{example}
c. $4x^4 - 1$
\[
4x^4 - 1 = (2x^2 + 1)(2x^2 - 1) = (2x^2 + 1)(\sqrt{2}x - 1)(\sqrt{2}x + 1)
\]
\end{example}

\begin{example}
d. $x^2 - y^2$
\[
x^2 - y^2 = (x + y)(x - y)
\]
\end{example}

\begin{example}
e. $(2x + 3)^2 - (x - 2)^2$
\[
(2x + 3)^2 - (x - 2)^2 = ((2x + 3) + (x - 2))((2x + 3) - (x - 2)) = (3x + 5)(x + 1)
\]
\end{example}

\begin{example}
f. $x^4 - y^4$
\[
x^4 - y^4 = (x^2 + y^2)(x^2 - y^2) = (x^2 + y^2)(x + y)(x - y)
\]
\end{example}

\end{lesson}
\subsection*{6. Factor completely each of the following:}

\begin{example}
a. $x^2 - 11x + 18$
\begin{align*}
x^2 - 11x + 18 &= (x - 9)(x - 2) \\
&= \text{Find two numbers that multiply to 18 and add up to -11.} \\
&= (-9)(-2) = 18 \text{ and } (-9) + (-2) = -11
\end{align*}
\end{example}

\begin{example}
b. $x^2 + 11x - 42$
\begin{align*}
x^2 + 11x - 42 &= (x + 14)(x - 3) \\
&= \text{Find two numbers that multiply to -42 and add up to 11.} \\
&= (14)(-3) = -42 \text{ and } (14) + (-3) = 11
\end{align*}
\end{example}

\begin{example}
c. $x^2 + 15x + 54$
\begin{align*}
x^2 + 15x + 54 &= (x + 18)(x + 3) \\
&= \text{Find two numbers that multiply to 54 and add up to 15.} \\
&= (18)(3) = 54 \text{ and } (18) + (3) = 15
\end{align*}
\end{example}

\begin{example}
d. $x^2 - 21x + 54$
\begin{align*}
x^2 - 21x + 54 &= (x - 18)(x - 3) \\
&= \text{Find two numbers that multiply to 54 and add up to -21.} \\
&= (-18)(-3) = 54 \text{ and } (-18) + (-3) = -21
\end{align*}
\end{example}

\begin{example}
e. $x^2 - 16x + 64$
\begin{align*}
x^2 - 16x + 64 &= (x - 8)^2 \\
&= \text{Recognize that it's a perfect square trinomial.}
\end{align*}
\end{example}

\begin{example}
f. $x^2 - 16x - 80$
\begin{align*}
x^2 - 16x - 80 &= (x - 20)(x + 4) \\
&= \text{Find two numbers that multiply to -80 and add up to -16.} \\
&= (-20)(4) = -80 \text{ and } (-20) + (4) = -16
\end{align*}
\end{example}

\begin{example}
g. $x^4 + 15x^2 + 50$
\begin{align*}
x^4 + 15x^2 + 50 &= (x^2 + 10)(x^2 + 5) \\
&= \text{Find two numbers that multiply to 50 and add up to 15.} \\
&= (10)(5) = 50 \text{ and } (10) + (5) = 15
\end{align*}
\end{example}

\begin{example}
h. $x^4 - 18x^2 + 81$
\begin{align*}
x^4 - 18x^2 + 81 &= (x^2 - 9)^2 \\
&= \text{Recognize the difference of squares.} \\
&= (x - 3)^2(x + 3)^2
\end{align*}
\end{example}

\begin{example}
i. $x^2 + 6xy - 7y^2$
\begin{align*}
x^2 + 6xy - 7y^2 &= (x + 7y)(x - y) \\
&= \text{Find two numbers that multiply to -7 and add up to 6.} \\
&= (7y)(-y) = -7y^2 \text{ and } (7y) + (-y) = 6y
\end{align*}
\end{example}

\newpage 
\subsection*{7. Factor completely each of the following:}

\begin{example}
a. $6x^2 - 23x - 18$
\begin{align*}
6x^2 - 23x - 18 &= (3x + 2)(2x - 9) \\
&= \text{Find two numbers that multiply to -36 and add up to -23.} \\
&= (3x)(-9) = -27x \text{ and } (3x) + (-9) = -23x
\end{align*}
\end{example}

\begin{example}
b. $12x^2 - 5x - 2$
\begin{align*}
12x^2 - 5x - 2 &= (4x - 1)(3x + 2) \\
&= \text{Find two numbers that multiply to -24 and add up to -5.} \\
&= (-1)(-2) = 2 \text{ and } (-1) + (2) = 1
\end{align*}
\end{example}

\begin{example}
c. $4x^2 + 17x + 4$
\begin{align*}
4x^2 + 17x + 4 &= (4x + 1)(x + 4) \\
&= \text{Find two numbers that multiply to 4 and add up to 17.} \\
&= (1)(4) = 4 \text{ and } (1) + (4) = 5
\end{align*}
\end{example}

\begin{example}
d. $9x^2 - 30x + 25$
\begin{align*}
9x^2 - 30x + 25 &= (3x - 5)^2 \\
&= \text{Recognize that it's a perfect square trinomial.}
\end{align*}
\end{example}

\begin{example}
e. $4x^4 - 3x^2 - 1$
\begin{align*}
4x^4 - 3x^2 - 1 &= (2x^2 - 1)(2x^2 + 1) \\
&= \text{Recognize the difference of squares.} \\
&= (2x - 1)(2x + 1)(2x^2 + 1)
\end{align*}
\end{example}

\begin{example}
f. $6x^2 - 12x - 18$
\begin{align*}
6x^2 - 12x - 18 &= 6(x - 3)(x + 1) \\
&= \text{Factor out the common factor.}
\end{align*}
\end{example}

\begin{example}
g. $2x^3 - 3x^2 + x$
\begin{align*}
2x^3 - 3x^2 + x &= x(2x^2 - 3x + 1) \\
&= x(2x - 1)(x - 1) \\
&= \text{Find two numbers that multiply to 2 and add up to -3.} \\
&= (2x)(-1) = -2x \text{ and } (2x) + (-1) = -x
\end{align*}
\end{example}

\newpage


\begin{example}
h. $4x^4 - 13x^2 + 9$
\begin{align*}
4x^4 - 13x^2 + 9 &= (2x^2 - 1)(2x^2 - 9) \\
&= (2x^2 - 1)(x - 3)(x + 3) \\
&= \text{Find two numbers that multiply to -9 and add up to -13.} \\
&= (-3)(3) = -9 \text{ and } (-3) + (3) = 0
\end{align*}
\end{example}

\begin{example}
i. $8x^4 - 31x^2 - 4$
\begin{align*}
8x^4 - 31x^2 - 4 &= (4x^2 - 1)(2x^2 + 4) \\
&= (2x - 1)(2x + 1)(2x^2 + 4) \\
&= \text{Recognize the difference of squares.}
\end{align*}
\end{example}

\begin{example}
j. $15x^2 - 29x - 14$
\begin{align*}
15x^2 - 29x - 14 &= (5x + 7)(3x - 2) \\
&= \text{Find two numbers that multiply to -14 and add up to -29.} \\
&= (7)(-2) = -14 \text{ and } (7) + (-2) = 5
\end{align*}
\end{example}

\begin{example}
k. $21x^2 - 29x + 10$
\begin{align*}
21x^2 - 29x + 10 &= (3x - 2)(7x - 5) \\
&= \text{Find two numbers that multiply to 10 and add up to -29.} \\
&= (-2)(-5) = 10 \text{ and } (-2) + (-5) = -7
\end{align*}
\end{example}

\begin{example}
l. $22x^2 + 43x - 2$
\begin{align*}
22x^2 + 43x - 2 &= (11x - 1)(2x + 2) \\
&= \text{Find two numbers that multiply to -22 and add up to 43.} \\
&= (-1)(-2) = 2 \text{ and } (-1) + (-2) = -3
\end{align*}
\end{example}
\newpage 
\subsection*{8. Factor completely each of the following:}

\begin{example}
a. $x^3 - 4x^2 + 3x - 12$
\begin{align*}
x^3 - 4x^2 + 3x - 12 &= (x - 3)(x - 2)(x + 2) \\
&= \text{Factor by grouping.}
\end{align*}
\end{example}

\begin{example}
b. $2x^3 - 6x^2 - 3x + 9$
\begin{align*}
2x^3 - 6x^2 - 3x + 9 &= 2(x - 3)(x - 1)(x + 3) \\
&= \text{Factor by grouping.}
\end{align*}
\end{example}

\begin{example}
c. $4x^3 + 8x^2 - x - 2$
\begin{align*}
4x^3 + 8x^2 - x - 2 &= 4(x - 1)(x + 1)(x + 2) \\
&= \text{Factor by grouping.}
\end{align*}
\end{example}

\begin{example}
d. $2x^3 - 6x^2 + 10x - 30$
\begin{align*}
2x^3 - 6x^2 + 10x - 30 &= 2(x - 3)(x - 5)(x + 1) \\
&= \text{Factor by grouping.}
\end{align*}
\end{example}
\newpage 
\begin{example}
e. $3x^5 - 12x^3 - x^2 + 4$
\begin{align*}
3x^5 - 12x^3 - x^2 + 4 &= (x - 2)(x + 2)(x - 1)(x^2 + x + 2) \\
&= \text{Factor by grouping.}
\end{align*}
\end{example}

\begin{example}
f. $2x^4 - 4x^3 - 8x^2 + 16x$
\begin{align*}
2x^4 - 4x^3 - 8x^2 + 16x &= 2x(x - 2)(x + 2)(x^2 - 4) \\
&= \text{Factor by grouping.}
\end{align*}
\end{example}


\newpage



\subsection*{9. Factor completely each of the following:}

\begin{example}
a. $x^2 - 13x + 22$
\begin{align*}
x^2 - 13x + 22 &= (x - 11)(x - 2) \\
&= \text{Find two numbers that multiply to 22 and add up to -13.} \\
&= (-11)(-2) = 22 \text{ and } (-11) + (-2) = -13
\end{align*}
\end{example}

\begin{example}
b. $3x^2 - 9x$
\begin{align*}
3x^2 - 9x &= 3x(x - 3) \\
&= \text{Factor out the common factor.}
\end{align*}
\end{example}

\begin{example}
c. $6x^2 + 11x - 7$
\begin{align*}
6x^2 + 11x - 7 &= (2x - 1)(3x + 7) \\
&= \text{Find two numbers that multiply to -7 and add up to 11.} \\
&= (-1)(-7) = 7 \text{ and } (-1) + (7) = 6
\end{align*}
\end{example}

\begin{example}
d. $10x^3 - 21x^2 + 8x$
\begin{align*}
10x^3 - 21x^2 + 8x &= x(2x - 1)(5x - 8) \\
&= \text{Factor out the common factor.}
\end{align*}
\end{example}

\begin{example}
e. $x^3 - 3x^2 + 4x - 12$
\begin{align*}
x^3 - 3x^2 + 4x - 12 &= (x - 4)(x^2 + x + 3) \\
&= \text{Find two numbers that multiply to -12 and add up to -3.} \\
&= (-4)(3) = -12 \text{ and } (-4) + (3) = -1
\end{align*}
\end{example}

\begin{example}
f. $25x^2 - 49y^2$
\begin{align*}
25x^2 - 49y^2 &= (5x - 7y)(5x + 7y) \\
&= \text{Factor using the difference of squares.}
\end{align*}
\end{example}

\begin{example}
g. $4x^2 + 20x + 25$
\begin{align*}
4x^2 + 20x + 25 &= (2x + 5)(2x + 5) \\
&= \text{Factor using the difference of squares.}
\end{align*}
\end{example}

\begin{example}
h. $5x^2 - 30x + 45$
\begin{align*}
5x^2 - 30x + 45 &= 5(x - 3)(x - 3) \\
&= \text{Find two numbers that multiply to 45 and add up to -30.} \\
&= (-3)(-3) = 9 \text{ and } (-3) + (-3) = -6
\end{align*}
\end{example}
\newpage 
\begin{example}
i. $4x^2 + 19x - 5$
\begin{align*}
4x^2 + 19x - 5 &= (4x - 1)(x + 5) \\
&= \text{Find two numbers that multiply to -5 and add up to 19.} \\
&= (-1)(-5) = 5 \text{ and } (-1) + (5) = 4
\end{align*}
\end{example}
\begin{example}
j. $100a^2 - 36b^2$
\begin{align*}
100a^2 - 36b^2 &= (10a)^2 - (6b)^2 \\
&= (10a + 6b)(10a - 6b) \\
&= 2(5a + 3b) \cdot 2(5a - 3b) \\
&= 4(5a + 3b)(5a - 3b)
\end{align*}
\end{example}

\begin{example}
k. $4x^4 + 28x^2 + 49$
\begin{align*}
4x^4 + 28x^2 + 49 &= (2x^2 + 7)^2 \\
&= \text{Recognize that it's a perfect square trinomial.}
\end{align*}
\end{example}

\begin{example}
l. $12x^2 + 8x + 28$
\begin{align*}
12x^2 + 8x + 28 &= 4(3x^2 + 2x + 7) \\
&= \text{Factor out the common factor.}
\end{align*}
\end{example}
\newpage
\begin{lesson}{PART C}

\subsection*{10. Use difference of squares factoring to evaluate each of the following:}

\begin{example}
a. $51^2 - 49^2$
\begin{align*}
51^2 - 49^2 &= (51 + 49)(51 - 49) \\
&= 100 \cdot 2 \\
&= 200
\end{align*}
\end{example}

\begin{example}
b. $27^2 - 23^2$
\begin{align*}
27^2 - 23^2 &= (27 + 23)(27 - 23) \\
&= 50 \cdot 4 \\
&= 200
\end{align*}
\end{example}

\begin{example}
c. $121^2 - 111^2$
\begin{align*}
121^2 - 111^2 &= (121 + 111)(121 - 111) \\
&= 232 \cdot 10 \\
&= 2320
\end{align*}
\end{example}
\newpage
\begin{example}
d. $10000^2 - 9999^2$
\begin{align*}
10000^2 - 9999^2 &= (10000 + 9999)(10000 - 9999) \\
&= 19999 \cdot 1 \\
&= 19999
\end{align*}
\end{example}


\subsection*{11. Factor completely each of the following:}

\begin{example}
a. $(x+2)^2 - 3(x+2) + 2$
\begin{align*}
(x+2)^2 - 3(x+2) + 2 &= (x+2 - 1)(x+2 - 2) \\
&= (x+1)(x) \\
&= x(x+1)
\end{align*}
\end{example}

\begin{example}
b. $(x-1)^4 - 1$
\begin{align*}
(x-1)^4 - 1 &= \left((x-1)^2 + 1\right)\left((x-1)^2 - 1\right) \\
&= \left((x-1 + 1)(x-1 - 1) + 1\right)\left((x-1 + 1)(x-1 - 1) - 1\right) \\
&= \left(x(x-2) + 1\right)\left(x(x-2) - 1\right) \\
&= (x^2 - 2x + 1)(x^2 - 2x - 1)
\end{align*}
\end{example}
\newpage
\begin{example}
c. $x^2 - 8x + 16 - 4y^2$
\begin{align*}
x^2 - 8x + 16 - 4y^2 &= (x^2 - 8x + 16) - 4y^2 \\
&= (x - 4)^2 - (2y)^2 \\
&= (x - 4 + 2y)(x - 4 - 2y) \\
&= (x + 2y - 4)(x - 2y - 4)
\end{align*}
\end{example}


\subsection*{12. By considering factors, find the smallest value of $x$ such that $120x$ will be a perfect square.}

To make $120x$ a perfect square, we need to find a value of $x$ such that $120x$ can be expressed as the square of an integer.

The prime factorization of $120$ is $2^3 \cdot 3 \cdot 5$. To make it a perfect square, we need to have an even power of each prime factor. Therefore, we need to find the smallest value of $x$ such that each of these prime factors has an even power in $120x$.

For $2^3$ to have an even power, $x$ should be a multiple of $2^3$, which is $8$. For $3$ to have an even power, $x$ should be a multiple of $3$, and for $5$ to have an even power, $x$ should be a multiple of $5$.

So, the smallest value of $x$ that makes $120x$ a perfect square is the least common multiple (LCM) of $8$, $3$, and $5$, which is $120$. Therefore, $x = 120$ makes $120x$ a perfect square.
\end{lesson}
\end{document}
