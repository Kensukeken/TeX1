\documentclass{article}
\usepackage{amsmath, amssymb}
\usepackage{graphicx}
\usepackage{geometry}
\geometry{margin=1in}

\title{Sinusoidal Test Review}
\author{Your Name}
\date{\today}

\begin{document}

\maketitle

\section*{Properties of Periodic/Sinusoidal Functions}
Sinusoidal functions exhibit several key properties:
\begin{itemize}
    \item \textbf{Amplitude ($A$):} The maximum displacement from the mean position.
    \item \textbf{Period ($T$):} The length of one complete cycle.
    \item \textbf{Frequency ($f$):} The number of cycles per unit time, where $f = \frac{1}{T}$.
    \item \textbf{Phase Shift ($\phi$):} Horizontal shift of the graph.
\end{itemize}

\section*{Finding Equations from a Graph}
To find the equation of a sinusoidal function from a graph, consider the amplitude, period, phase shift, and vertical shift. For example, if a graph has an amplitude of $2$, a period of $4$, a phase shift of $-\frac{\pi}{2}$, and a vertical shift of $3$, the equation can be written as:
\[ f(x) = 2 \sin\left(\frac{2\pi}{4}(x + \frac{\pi}{2})\right) + 3 \]

\section*{Graphing}
To graph a sinusoidal function, follow these steps:
\begin{enumerate}
    \item Identify the amplitude, period, phase shift, and vertical shift.
    \item Use these values to sketch the basic shape of the sinusoidal function.
    \item Plot key points and complete the graph.
\end{enumerate}

\section*{Word Problem(s) - Ferris Wheel}
Consider a Ferris Wheel with a diameter of $50$ meters. The wheel makes one full rotation every $40$ seconds. The center of the wheel is $5$ meters above the ground. Answer the following:
\begin{enumerate}
    \item \textbf{Write the equation for the height of a rider as a function of time:} \\
    Let $h(t)$ represent the height of a rider at time $t$. The equation is given by:
    \[ h(t) = 25 \sin\left(\frac{2\pi}{40}t\right) + 5 \]

    \item \textbf{Graph the function and identify key features:} \\
    Use the equation to sketch the graph, highlighting amplitude, period, phase shift, and vertical shift.

    \item \textbf{Solve real-life problems related to the Ferris Wheel:} \\
    For example, find the height of a rider at $t = 15$ seconds by substituting $t = 15$ into the equation.
\end{enumerate}

\section*{Conclusion}
Reviewing these concepts will prepare you for the Sinusoidal Test tomorrow. Remember to practice solving problems and interpreting graphs. Good luck!

\end{document}
