\documentclass[12pt,a4paper]{article}
\usepackage{amsmath,amsfonts,amssymb}
\usepackage{graphicx}
\usepackage{enumitem}
\usepackage[dvipsnames]{xcolor}
\usepackage{pgfplots}
\usepackage{hyperref}
\usepackage{soul}
\usepackage{framed}
\usepackage{booktabs} 
\usepackage{tabularx}
\usepackage{array}

\definecolor{lessonbgcolor}{rgb}{0.9,0.9,1}
\definecolor{examplecolor}{rgb}{0.8,1,0.8}
\definecolor{noteboxcolor}{rgb}{1,0.8,0.8}
\newenvironment{lesson}[1]
  {\begin{framed}\colorbox{lessonbgcolor}{
  \parbox{\dimexpr\linewidth-2\fboxsep}{
  \textbf{#1}}}\end{framed}}
  
\newenvironment{example}
  {\begin{framed}\colorbox{examplecolor}{
  \parbox{\dimexpr\linewidth-2\fboxsep}{
  \textbf{Example:}}}}
  {\end{framed}}
\newenvironment{note}
  {\begin{framed}\colorbox{noteboxcolor}{
  \parbox{\dimexpr\linewidth-2\fboxsep}{
  \textbf{Note:}}}}
  {\end{framed}}

\begin{document}


\section*{Unit 2: Rational Expressions}

\begin{lesson}{Lesson 1 - Review of Exponent Rules}
\textcolor{blue}{
The exponent rules are foundational principles that dictate how terms with the same base can be combined.
\begin{enumerate}
\item \(a^m \times a^n = a^{m+n}\) 
\item \(\frac{a^m}{a^n} = a^{m-n}\)
\item \((a^m)^n = a^{m \times n}\)
\end{enumerate}
}
\begin{example}
1. Using Rule 1: \(2^3 \times 2^4 = 2^{3+4} = 2^7\) \\
2. Using Rule 2: \(\frac{5^7}{5^4} = 5^{7-4} = 5^3\) \\
3. Using Rule 3: \((3^2)^3 = 3^{2 \times 3} = 3^6\)
\end{example}
\end{lesson}

\begin{lesson}{Lesson 2 - Rational Exponents}
\textcolor{blue}{
Rational exponents refer to exponents that are fractions. They can often be represented as roots.
\[
a^{\frac{m}{n}} = \sqrt[n]{a^m}
\]
}
\begin{example}
1. Using the formula: \(9^{\frac{1}{2}} = \sqrt[2]{9} = 3\) \\
2. \(16^{\frac{1}{4}} = \sqrt[4]{16} = 2\) \\
3. \(8^{\frac{2}{3}} = \sqrt[3]{8^2} = 4\)
\end{example}
\end{lesson}
\newpage
\begin{lesson}{Lesson 3 - Simplifying, Multiplying and Dividing Rational Expressions}
\textcolor{blue}{
Rational expressions are fractions wherein either the numerator, the denominator, or both are polynomials.
\begin{enumerate}
\item To multiply: Multiply the numerators with each other and the denominators with each other.
\item To divide: Multiply the first fraction by the reciprocal of the second.
\end{enumerate}
}
\begin{example}
1. Multiplication: \(\frac{x}{y} \times \frac{z}{w} = \frac{x \times z}{y \times w}\) \\
2. Division: \(\frac{x}{y} \div \frac{z}{w} = \frac{x}{y} \times \frac{w}{z}\) \\
3. Simplifying: \(\frac{3x}{6y} = \frac{x}{2y}\) \textcolor{red}{Divided by 2}
\end{example}
\end{lesson}

\begin{lesson}{Lesson 4 - Adding and Subtracting Rational Expressions}
\textcolor{blue}{
To add or subtract rational expressions:
\begin{enumerate}
\item Find a common denominator.
\item Rewrite each fraction with that denominator.
\item Add or subtract the numerators.
\end{enumerate}
}
\begin{example}
1. \(\frac{a}{c} + \frac{b}{d} = \frac{ad + bc}{cd}\) given that \(cd\) is the common denominator. \\
2. \(\frac{3x}{x^2-1} + \frac{2x}{x^2+2x} = \frac{3x(x+2) + 2x(x-1)}{x^2-1}\) \\
3. \(\frac{5}{x+3} - \frac{2}{x-2} = \frac{5(x-2) - 2(x+3)}{(x+3)(x-2)}\)
\end{example}
\end{lesson}

\begin{center}
\begin{tabular}{l l}
\textbf{Exponent Rule Name} & \textbf{Property} \\
\hline
Product Rule & \(a^x a^y = a^{x+y}\) \\
Negative Exponent & \(a^{-x} = \frac{1}{a^x}\) \\
Quotient Rule & \(\frac{a^x}{a^y} = a^{x-y}\) \\
Power of Power & \((a^x)^y = a^{xy}\) \\
Distributivity & \((ab)^x = a^x b^x\) \\
Fractional Exponent & \(a^{\frac{x}{y}} = \sqrt[y]{a^x}\) \\
\end{tabular}
\end{center}


\section*{\hl{Factoring Review}}
Factoring is the process of expressing a polynomial as a product of simpler polynomials. This document will demonstrate how to factor polynomials with examples and step-by-step explanations.

\begin{lesson}{Factoring Basics}
    To factor a polynomial, we look for common factors and apply various factoring techniques. Here are some common factoring methods:
\begin{enumerate}
    \item Factoring out the greatest common factor (GCF).
    \item Factoring by grouping.
    \item Factoring the difference of squares.
    \item Factoring trinomials of the form $ax^2 + bx + c$.
    \item Factoring special forms like the sum or difference of cubes.
\end{enumerate}

\end{lesson}


\begin{lesson}{Examples}
\end{lesson}
\begin{example}
Factoring the GCF.\\
Factor the polynomial $6x^2 + 12x$.
\begin{align*}
6x^2 + 12x &= 6x(x + 2) \quad \text{(Factor out the GCF, 6x)}
\end{align*}
\end{example}

\begin{example}
Factoring by Grouping.\\
Factor the polynomial $x^3 - x^2 + 4x - 4$.
\begin{align*}
x^3 - x^2 + 4x - 4 &= (x^3 - x^2) + (4x - 4) \quad \text{(Group the terms)} \\
&= x^2(x - 1) + 4(x - 1) \quad \text{(Factor out common factors)} \\
&= (x^2 + 4)(x - 1) \quad \text{(Factor further if possible)}
\end{align*}
\end{example}
\begin{example}
Factoring the Difference of Squares.\\
Factor the polynomial $9y^2 - 16z^2$.
\begin{align*}
9y^2 - 16z^2 &= (3y)^2 - (4z)^2 \quad \text{(Recognize it as a difference of squares)} \\
&= (3y + 4z)(3y - 4z) \quad \text{(Apply the difference of squares formula)}
\end{align*}
\end{example}
\begin{example}
Factoring a Trinomial.\\
Factor the trinomial $x^2 + 5x + 6$.

\begin{align*}
x^2 + 5x + 6 &= (x + 2)(x + 3) \quad \text{(Find two numbers that multiply to 6 and add up to 5)}
\end{align*}
\end{example}
\begin{example}
Factoring the Sum of Cubes.\\
Factor the polynomial $x^3 + 8$.
\begin{align*}
x^3 + 8 &= (x + 2)(x^2 - 2x + 4) \quad \text{(Recognize it as a sum of cubes)} \\
&= (x + 2)(x - 1 + 2i)(x - 1 - 2i) \quad \text{(Factor the quadratic using the quadratic formula)}
\end{align*}
\end{example}
\end{document}
