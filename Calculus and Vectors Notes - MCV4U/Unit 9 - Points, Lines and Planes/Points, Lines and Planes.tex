\documentclass{article}
% PACKAGES
\usepackage[english]{babel}
\usepackage{graphicx} % Required for inserting images
\usepackage[most]{tcolorbox}
\usepackage{lmodern}
\usepackage{titlepic}
\usepackage{pdfpages}
\usepackage{tcolorbox}
\usepackage{amsmath}
\usepackage{pgfplots}
\usepackage{xcolor}
\usepackage{tikz}
\usepackage{color,soul}
\usepackage{enumerate}
\usepackage{enumitem}
\usepackage{cancel}
\usepackage{hyperref} 
\usepackage{tikzsymbols}
\usepackage{fontawesome5}
\usepackage[export]{adjustbox}
\usepackage{amssymb}
\usepackage{tikz,lipsum,lmodern}
\usepackage{booktabs}
\usepackage{tikz-3dplot}
\usepackage{circuitikz}
\usetikzlibrary{shadings}

% COLOURS
\definecolor{Orchid}{RGB}{218, 112, 214}
\definecolor{snow}{rgb}{1.0, 0.98, 0.98}
\definecolor{mordantred19}{rgb}{0.68, 0.05, 0.0}
\definecolor{mistyrose}{rgb}{1.0, 0.89, 0.88}
\definecolor{nadeshikopink}{rgb}{0.96, 0.68, 0.78}
\definecolor{cadmiumgreen}{rgb}{0.0, 0.42, 0.24}
\definecolor{OliveGreen}{RGB}{85, 107, 47}
\definecolor{RoyalPurple}{RGB}{120, 81, 169}
\definecolor{NavyBlue}{RGB}{0, 0, 128}
\definecolor{CornflowerBlue}{RGB}{100, 149, 237}
\definecolor{Cerulean}{RGB}{0, 123, 167}
\definecolor{DarkOrchid}{RGB}{153, 50, 204}
\definecolor{Salmon}{rgb}{1.0, 0.55, 0.41}
\definecolor{FireBrick}{rgb}{0.7, 0.13, 0.13}

% Define the new tcolorbox style
\newtcolorbox{mybeamer}[2][]{beamer,
    colback=Salmon!50!white, 
    colframe=FireBrick!75!black, 
    adjusted title={#2}, 
    #1
}

\tcbset{skin=enhanced,fonttitle=\bfseries,
frame style={upper left=blue,upper right=red,lower left=yellow,lower right=green},
interior style={white,opacity=0.5},
segmentation style={black,solid,opacity=0.2,line width=1pt}}



\usetikzlibrary{calc}

\title{MCV4U - Calculus and Vectors [Chapter 9]}
\author{Kensukeken}
\date{June 4th, 2024}

\begin{document}
\maketitle

\tableofcontents
\newpage
\section{Unit 9}
\subsection{Intersection of a Line with a Plane or Two Lines Intersecting}
\subsubsection{Part 1: Intersection of a Line with a Plane: 3 Cases}
\begin{enumerate}
    \item[1.] The line \( L \) intersects the plane \( \pi \) at exactly one point, \( P \).
    \item[2.] The line does not intersect the plane, so it is parallel to the plane. There are no points of intersection.
    \item[3.] The line \( L \) lies on the plane \( \pi \). Every point on \( L \) intersects the plane. There are an infinite number of points of intersection.
\end{enumerate}

\begin{figure}[!ht]
\centering
\resizebox{1\textwidth}{!}{%
\begin{circuitikz}
\tikzstyle{every node}=[font=\LARGE]

% First diagram
\draw [short] (0.25,9.75) -- (2.25,12.5);
\draw [short] (2.25,12.5) -- (7,12.5);
\draw [short] (0.25,9.75) -- (5.25,9.75);
\draw [short] (7,12.5) -- (5.25,9.75);
\draw [color={rgb,255:red,0; green,251; blue,255}, <->, >=Stealth] (3.5,11.25) -- (5,13.75);
\draw [color={rgb,255:red,0; green,255; blue,213}, dashed] (3.5,11.25) -- (2.5,9.5);
\draw [color={rgb,255:red,0; green,255; blue,213}, ->, >=Stealth] (2.5,9.5) -- (1.75,8.25);
\node [font=\LARGE] at (4.5,13.5) {L};
\node [font=\LARGE] at (3,11.5) {P};
\node [font=\LARGE] at (6.25,12.25) {$\pi$};

% Second diagram
\draw [short] (9.25,10.25) -- (10.5,12.25);
\draw [short] (10.5,12.25) -- (14.5,12.25);
\draw [short] (14.5,12.25) -- (13.5,10.25);
\draw [short] (9.25,10.25) -- (13.5,10.25);
\draw [color={rgb,255:red,247; green,2; blue,2}, <->, >=Stealth] (9,13.25) -- (14.5,13.25);
\draw [color={rgb,255:red,0; green,251; blue,255}, ->, >=Stealth] (11,11.25) -- (11,14.75);
\draw [color={rgb,255:red,0; green,255; blue,213}, dashed] (11,11.25) -- (11,10.25);
\draw [color={rgb,255:red,0; green,251; blue,255}, short] (11,10.25) -- (11,9.25);
\draw [color={rgb,255:red,255; green,0; blue,0}, short] (11,13) -- (11.25,13);
\draw [color={rgb,255:red,255; green,0; blue,0}, short] (11.25,13.25) -- (11.25,13);
\node [font=\normalsize] at (12.75,13.5) {L, direction vector m};
\node [font=\large] at (11,15) {n};
\node [font=\large] at (11,15.25) {$\rightarrow$};

% Third diagram
\draw [short] (17.25,10) -- (18.75,12.25);
\draw [short] (18.75,12.25) -- (22.75,12.25);
\draw [short] (17.25,10) -- (21.25,10);
\draw [short] (22.75,12.25) -- (21.25,10);
\draw [color={rgb,255:red,0; green,255; blue,238}, <->, >=Stealth] (18,10.25) -- (21.5,11.75);
\node [font=\normalsize] at (19.75,11.5) {L};
\node [font=\normalsize] at (22,12) {$\pi$};
\node [font=\normalsize] at (13.75,12) {$\pi$};
\node [font=\small] at (14,13.75) {$\rightarrow$};

\end{circuitikz}
}
\label{fig:intersection_cases}
\end{figure}


\subsubsection{Procedure For Determining intersection(s) of a Line with a Plane(if any exist)}
\begin{tcolorbox}
\begin{enumerate}
    \item[1.] Set the parametric expressions for x,y and z into the Cartesian equation $Ax+Bx+Cz+D=0$.
    \item[2.] Let's assume the parameter is t. Solve the equation we get in part 1 above. If we get to 0t=non-zero then there are no solutions (because it is impossible for 0 a times a real number to equal to a non-zero quantity). Therefore there are no points of intersection, and we have a situation like case 2 on the previous slide(the line L is parallel to, but not on, the plane $\pi$)
    \item[3.] If we get 0t=0 then there are an infinite number of solutions, meaning that we have a situation like case 3 on the previous slide(the line L is on the plane $\pi$)
    \item[4.] If we get a non-zero quantity times t equalling a real number, then we have one solution, and we have a situation like case 1 on the previous slide(the line L intersects the plane $\pi$ at a single point)
\end{enumerate}
\end{tcolorbox}
\subsubsection*{Example}
Determine the point(s) of intersection, if any exist, between the line $L:\vec{r}=\overrightarrow{(3,-2,1)}+s\overrightarrow{(14,-5,-3)}$ and the plane $\pi:x+y+3z-4=0$
\subsubsection*{Solution}
\begin{align*}
    x&=3+14s\\
    y&=-2-5s\\
    z&=1-3s
\end{align*}
Plug in plane
\[
    3+14s-2-5s+3(1-3s)-4=0
\]

\[
    3+14s-2-5s+3-9s-4=0
\]
$0s=0$ infinite solutions. Therefore, infinite PTS of intersection 

\subsubsection{Part 2: The intersection of Two Lines (4 Cases)}
\begin{enumerate}
    \item[1.] Two lines intersect at a single point if
    \begin{itemize}
        \item The direction vectors are not collinear, and
        \item We algebraically determine a point of intersection.
    \end{itemize}
    \item[2.] Two lines are coincident if
    \begin{itemize}
        \item The direction vectors are collinear, and
        \item We algebraically determine a point of intersection.
    \end{itemize}
    \item[3.] Two lines are parallel \& non-coincident (no points of intersection) if
    \begin{itemize}
        \item The direction vectors are collinear, and
        \item We determine that one point of the lines is not on the other line.
    \end{itemize}
    \item[4.] Two lines are skew (no points of intersection) if
    \begin{itemize}
        \item The direction vectors are not collinear, and
        \item We algebraically determine that there are no points of intersection.
    \end{itemize}
\end{enumerate}
\vspace{2em}
\begin{figure}[!ht]
\centering
\resizebox{1\textwidth}{!}{%
\begin{circuitikz}
\tikzstyle{every node}=[font=\LARGE]
\draw [->, >=Stealth] (4.25,12.75) -- (2.25,10.75);
\draw [->, >=Stealth] (4.25,12.75) -- (4.25,16);
\draw [->, >=Stealth] (4.25,12.75) -- (8,12.75);
\draw [<->, >=Stealth] (2.5,13.25) -- (5.75,14.5);
\draw [<->, >=Stealth] (3,14.75) -- (5.5,11.75);
\node [font=\large] at (4,14.25) {A};
\node [font=\normalsize] at (4.25,16.25) {z};
\node [font=\normalsize] at (2.25,10.5) {x};
\node [font=\normalsize] at (8.25,12.75) {y};
\draw [->, >=Stealth] (14.5,13) -- (13,11.25);
\draw [->, >=Stealth] (14.5,13) -- (14.5,15.75);
\draw [->, >=Stealth] (14.5,13) -- (17.75,13);
\draw [<->, >=Stealth] (13,13.5) -- (16.75,14.75);
\node [font=\normalsize] at (14.5,16) {z};
\node [font=\normalsize] at (13,11) {x};
\node [font=\normalsize] at (18,13) {y};
\draw [->, >=Stealth] (4.25,4.75) -- (2.5,3);
\draw [->, >=Stealth] (4.25,4.75) -- (4.25,7.25);
\draw [->, >=Stealth] (4.25,4.75) -- (7,4.75);
\draw [<->, >=Stealth] (2.25,4) -- (6.25,5.5);

\draw [line width=0.5pt, <->, >=Stealth] (2.25,4.25) -- (6.25,5.75);
\draw [line width=0.5pt, ->, >=Stealth] (14.25,4.5) -- (12.75,2.75);
\draw [line width=0.5pt, ->, >=Stealth] (14.25,4.5) -- (17.75,4.5);
\draw [->, >=Stealth] (14.25,4.5) -- (14.25,7.25);
\draw [<->, >=Stealth] (12.5,5.75) -- (16.5,6.5);
\draw [<->, >=Stealth] (13.5,5.5) -- (16.75,3.75);
\node [font=\normalsize] at (4.25,7.5) {z};
\node [font=\normalsize] at (7.25,4.75) {y};
\node [font=\normalsize] at (2.5,2.75) {x};
\node [font=\normalsize] at (12.75,2.5) {x};
\node [font=\normalsize] at (14.25,7.5) {z};
\node [font=\normalsize] at (18,4.5) {y};
\node [font=\LARGE] at (10,16.25) {Intersecting Lines};
\node [font=\LARGE] at (9.75,7.75) {Non-intersecting Lines};
\end{circuitikz}
}%

\label{fig:my_label}
\end{figure}

\vspace{4em}

\subsubsection*{Example}
Determine the point(s) of intersection, if any, of the lines \( L_1: \vec{r}=\overrightarrow{(-4,-1,-6)}+s\overrightarrow{(3,2,-5)}, \, s\in \mathbb{R} \) and \( L_2: \vec{r}=\overrightarrow{(8,7,-26)}+t\overrightarrow{(-6,-4,10)}, \, t \in \mathbb{R} \)

\subsubsection*{Solution}
Parallel lines \\
\(\therefore\) check if \(\overrightarrow{(-4,-1,-6)}\) is on \( L_2 \)
\begin{align*}
    -4 &= 8 - 6t \implies t = 2 \\
    -1 &= 7 - 4t \implies t = 2 \\
    -6 &= -26 + 10t \implies t = 2
\end{align*}
The lines intersect at a point since the lines are parallel and intersect at a point. \(\therefore\) The lines are parallel and coincident.

\subsection{System of Equations}
A linear system of equations can have zero, one, or an infinite number of solutions. Think of two lines. Those of two lines might be parallel and not intersect (no solution), or they might be parallel and intersect at one point (one solution), or they might be parallel and consistent (an infinite number of solutions).
\subsubsection{Terms to Know}
An \textbf{inconsistent system} is a system of equations that has zero solutions.\\
A \textbf{consistent system} is a system of equations that has one solution or an infinite number of solutions.\\\\
Two system of equations are defined as \textbf{equivalent systems of equations} if every solution to one system is also a solution to the second system of equations, and vice versa.

\subsubsection{Elementary Row Operations }
\begin{enumerate}
    \item[1.] Multiply an equation by a nonzero constant.
    \item[2.] Interchange any pair of equations. 
    \item[3.] Add a non-zero multiple of one equations to replace the second equation.
\end{enumerate}
\subsubsection{Matrices}
A matrix(plural matrices) is very powerful tool in mathematics. For now, we will introduce matrices as a way to organize our solution to solve a system of equations.\\
Matrices have many more uses than this, and there are entire university math courses dedicated to them.
\subsubsection*{Example}
Solve the following system of equations using a matrix and elementary row operations.
$$2x+y=-9$$
$$x+2x=-6$$
\subsubsection*{Solution}
$ \left[\begin{array}{ll|l}
2 & 1 & -9 \\
1 & 2 & -6
\end{array}\right] \quad \rightarrow_{2\times R_2}$
$ \left[\begin{array}{ll|l}
2 & 1 & -9 \\
2 & 4 & -12
\end{array}\right] \quad \rightarrow_{R_1-R_2}$
$ \left[\begin{array}{ll|l}
2 & 1 & -9 \\
0 & -3 & 3
\end{array}\right]$
$$
\begin{aligned}
-3y &= 3 & \implies y &= -1 \\
2x - 1 &= -9 & \implies 2x &= -8 \\
& & \implies x &= -4\\
\therefore \text{One POT } (-4,-1) \text{consistent system.}
\end{aligned}
$$
\subsubsection*{Example 2}
Solve the following system of equations using matrix and elementary row operations.
$$3x+4y=8$$
$$6x+8y=15$$
\subsubsection*{Solution}
$ \left[\begin{array}{ll|l}
3 & 4 & 8 \\
6 & 8 & 15
\end{array}\right] \quad \rightarrow_{R_1\times 2}$
$ \left[\begin{array}{ll|l}
6 & 8 & 16 \\
6 & 8 & 15
\end{array}\right] \quad \rightarrow_{R_1-R_2}$
$ \left[\begin{array}{ll|l}
6 & 8 & 16 \\
0 & 0 & 1
\end{array}\right] $\\\\
$\therefore 0x+0y=1$ not possible. No solutions. Inconsistent system. 

\subsubsection*{Example 3}
Solve the following system of equations using matrix and elementary row operations.
$$5x-2y=8$$
$$6x+8y=15$$
\subsubsection*{Solution}
$ \left[\begin{array}{ll|l}
5 & -2 & 11 \\
55 & -22 & 121
\end{array}\right] \quad \rightarrow_{11 \times R_1}$
$ \left[\begin{array}{ll|l}
55 & -22 & 121 \\
55 & -22 & 121
\end{array}\right] \quad \rightarrow_{R_1 - R_1}$
$ \left[\begin{array}{ll|l}
55 & -22 & 121 \\
0 & 0 & 0
\end{array}\right] $\\\\
$0x+0y=0$ has infinite \# of POT's. Therefore, consistent system.
\subsubsection{Examples with 3 Equations and 3 Unknowns}
\subsubsection*{Example 4}
Solve the following system of equations.
\begin{align*}
    3x-4y+2z+5=0\\
    2x+y-7z+32=0\\
    2x-3y+8z-17=0
\end{align*}
\subsubsection*{Solution}
$ \left[\begin{array}{lll|l}
3 & -4 & 2 & -5\\
2 & 1 & -7 & -32\\
2 & -3 & 8 & 17
\end{array}\right] \quad \rightarrow_{2 \times R_1}$
$ \left[\begin{array}{lll|l}
6 & -8 & 4 & -10\\
6 & 3 & -21 & -96\\
6 & -9 & 24 & 51
\end{array}\right] \quad \rightarrow_{R_1-R_2 } \quad _{R_1-R_3}$
$ \left[\begin{array}{lll|l}
6 & -8 & 4 & -10\\
0 & -11 & 25 & 86\\
0 & 1 & 20 & -61
\end{array}\right] \quad \rightarrow_{11 \times R_3}$
$ \left[\begin{array}{lll|l}
6 & -8 & 4 & -10\\
0 & -11 & 25 & 86\\
0 & 11 & -220 & -671
\end{array}\right] \quad \rightarrow_{R_2+R_3}$
$$\left[\begin{array}{lll|l}
6 & -8 & 4 & -10\\
0 & -11 & 25 & 86\\
0 & 0 & -195 & -585
\end{array}\right] $$
\begin{align*}
    \text{from } R_3 \quad -195z&=-585 &&6x-8x+4z=-10\\
    z&=3 && 6x-8y(-1)+4(3)=-10\\
    -11y+25z&=8 && 6x=-30\\
    11y+25(3)&=86 && x=-5\\
    -11y&=11 \\
    y&=-1
\end{align*}
$\therefore$ One POT at $(-5,-1,3)$
\subsection{Intersection of Two Planes: 3 Cases}
There are 3 cases when it comes to the intersection of 2 planes. \\
Case 1: Two planes intersect along a line (infinite points of intersection)\\
Case 2: Two parallel planes(no points of intersection)\\
Case 3: Two coincident planes (infinite points of intersection)

\begin{itemize}
    \item Is it not possible for two planes to intersect at only one point. There are either zero points of intersection (like in case 2) or there are an infinite number of intersection (a line in case 1 or a plane like in case 3).
\end{itemize}
\begin{figure}[!ht]
\centering
\resizebox{1\textwidth}{!}{%
\begin{circuitikz}
\tikzstyle{every node}=[font=\small]
\draw [short] (2.25,12) -- (6,12);
\draw [short] (2.25,12) -- (1.5,11);
\draw [short] (1.5,11) -- (5.25,11);
\draw [short] (6,12) -- (5.25,11);
\draw [short] (3.5,12.75) -- (4.25,13.25);
\draw [short] (3.5,12.75) -- (3.5,10);
\draw [short] (3.5,10) -- (4.25,10.75);
\draw [short] (4.25,13.25) -- (4.25,10.75);
\draw [dashed] (4.25,12) -- (3.5,11);
\node [font=\small] at (3.75,11.75) {L};
\node [font=\small] at (2,11.25) {$\pi_2$};
\node [font=\small] at (3.75,12.75) {$\pi_1$};
\node [font=\small] at (4,13.75) {Case 1: Two Planes intersecting along a Line};
\draw [short] (8.5,12.25) -- (9.25,13);
\draw [short] (9.25,13) -- (11.75,13);
\draw [short] (8.5,12.25) -- (11,12.25);
\draw [short] (11.75,13) -- (11,12.25);
\draw [short] (9,11.75) -- (8.25,11);
\draw [short] (9,11.75) -- (11.5,11.75);
\draw [short] (8.25,11) -- (10.75,11);
\draw [short] (11.5,11.75) -- (10.75,11);
\node [font=\small] at (9,12.5) {$\pi_1$};
\node [font=\small] at (8.75,11.25) {$\pi_2$};
\node [font=\small] at (10,13.75) {Case 2: Two parallel Planes};
\draw [short] (13.75,11.75) -- (14.25,12.5);
\draw [short] (14.25,12.5) -- (17.5,12.5);
\draw [short] (17.5,12.5) -- (17,11.75);
\draw [short] (13.75,11.75) -- (17,11.75);
\node [font=\small] at (14.5,12.25) {$\pi_1 \pi_2$};
\node [font=\small] at (15.75,13.75) {$Case 3: Two Coincident Planes$};
\end{circuitikz}
}%

\label{fig:my_label}
\end{figure}

\subsubsection{Method for Determining the intersection of Two Planes}
\begin{enumerate}
    \item [1.] Make sure that the planes are in Cartesian form and compare the normal vectors.
    \item [2.] If the normal vectors are collinear (i.e., if $\overrightarrow{n_1}$ is a scalar multiple of $\overrightarrow{n_2}$ ) then one of the following two situations is true.
    \begin{enumerate}
    \item [a.] Perhaps the planes are parallel and coincident creating inifinite points of intersection. This is true if the scalar multiple between the normal vectors of the two planes is also the scalar multiple between the constant terms (i.e., the $D$ values of the two planes)
    \item[b.] Perhaps the planes are parallel and non-coincident, creating zero points of intersection. This is true if the scalar multiple between the normal vectors of the two planes is not the scalar multiple between the constant terms (i.e., the D values of the two planes)
    \end{enumerate}

\item[3.] If the normal vectors are not collinear (i.e., if $\overrightarrow{n_1}$ is not a scalar multiple of $\overrightarrow{n_2}$ ) then the two planes intersect along a line and there are infinite points of intersection.
\begin{enumerate}
    \item[a.] Make a matrix with the Cartesian equations of the planes
    \item[b.] Use elementary row operations to get a zero in one of the three left-hand columns of one of the rows
    \item[c.] Using that row, sub in a value for one of the remaining columns, and determine the value for the other column
    \item[d.] Then, sub both of those values in to the other row to determine a value for the third column. You now have a position vector.
    \item[e.] The direction vector is the cross product of the normal vectors of the planes.
    \item[f.] You can now state the equation of the line in vector form.
\end{enumerate}
\end{enumerate}

\subsubsection*{Example 1}
Determine whether the following planes intersect. If they do intersect, state how they intersect, and include the equation of the line of intersection if necessary.
$\pi_1: 2x-3y+7z-25=0 \quad \text{ and } \quad \pi_2:4x-6y+14z-50=0$
\subsubsection*{Solution}
Multiply both sides of $\pi_1$ by 2 $4x-6y+14z-50=0$. \\
Therefore, the planes are parallel and coincident (infinite POT'S). It is consistent  system.

\subsubsection*{Example 2}
Determine whether the following planes intersect. If they do intersect, state how they intersect, and include the equation of the line of intersection if necessary.
$\pi_1: 9x-y+4z-16=0 \quad \text{ and } \quad \pi_2:-18x+2y-8z+8=0$
\subsubsection*{Solution}
$\vec{n}_1=(9,-1,4) \quad \vec{n}_2=(-18,2,-8)$. $\therefore $ planes are parallel.\\
Multiply both side of $\pi_1$ by -2.
$$-18x+2y-8z+32$$
$\therefore$ parallel and non-coincident. 0 POT'S incoincident system.
\end{document}